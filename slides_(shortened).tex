%%%%%%%%%%%%%%%%%%%%%%%%%%%%%%%%%%%%%%%%%
% Beamer Presentation
% LaTeX Template
% Version 1.0 (10/11/12)
%
% This template has been downloaded from:
% http://www.LaTeXTemplates.com
%
% License:
% CC BY-NC-SA 3.0 (http://creativecommons.org/licenses/by-nc-sa/3.0/)
%
%%%%%%%%%%%%%%%%%%%%%%%%%%%%%%%%%%%%%%%%%

%----------------------------------------------------------------------------------------
%	PACKAGES AND THEMES
%----------------------------------------------------------------------------------------

\documentclass
		[
			xcolor = table,			% allows the coloring of whole tabular by changing the color before the definition of the tabular
			usenames,			% allows the usage of the names of the 16 default colors
			dvipsnames,			% allows the usage of the names of 64 additional colors
			svgnames,			% allows the usage of the names of ca. 150 additional colors
			x11names				% allows the usage of the names of ca. 300 additional colors
			final
		]{beamer}					% setting of the document class


%%%%%%%%%%%%%%%%%%%%%%%%%%%%%%%%%%%%%%%%%%%%%%%
%%%%%%%%%%%%%%%%%%%%%%%%%%%%%%%%%%%%%%%%%%%%%%%
%%%%%%%%%%%%%%%%%%%%%%%%%%%%%%%%%%%%%%%%%%%%%%%
% Defines a toggle to switch between grayscale and colored output:
%%%%%%%%%%%%%%%%%%%%%%%%%%%%%%%%%%%%%%%%%%%%%%% begindefinition
\usepackage{etoolbox}			% allows the usage of different 

\newtoggle{bwmode}
%%%%%%%%%%%%%%%%%%%%%%%%%%%%%%%%%%%%%%%%%%%%%%% enddefinition

%%%%%%%%%%%%%%%%%%%%%%%%%%%%%%%%%%%%%%%%%%%%%%%
%%%%%%%%%%%%%%%%%%%%%%%%%%%%%%%%%%%%%%%%%%%%%%%
%%%%%%%%%%%%%%%%%%%%%%%%%%%%%%%%%%%%%%%%%%%%%%%
% Redefines \thefootnote to use numbering starting from 0:
%%%%%%%%%%%%%%%%%%%%%%%%%%%%%%%%%%%%%%%%%%%%%%% begindefinition
\newcounter{indianfoot}
\newcommand{\useindianfootnotes}{
	\renewcommand{\thefootnote}{%
		\setcounter{indianfoot}{0}%
		\addtocounter{indianfoot}{\value{footnote}}%
		\arabic{indianfoot}}%
}
%%%%%%%%%%%%%%%%%%%%%%%%%%%%%%%%%%%%%%%%%%%%%%% enddefinition

%%%%%%%%%%%%%%%%%%%%%%%%%%%%%%%%%%%%%%%%%%%%%%%
%%%%%%%%%%%%%%%%%%%%%%%%%%%%%%%%%%%%%%%%%%%%%%%
%%%%%%%%%%%%%%%%%%%%%%%%%%%%%%%%%%%%%%%%%%%%%%%
% Redefines \thempfootnote to use numbering starting from 1 (should be 0 but buggy):
%%%%%%%%%%%%%%%%%%%%%%%%%%%%%%%%%%%%%%%%%%%%%%% begindefinition
\newcounter{indianmpfoot}
\newcommand{\useindianmpfootnotes}{
	\renewcommand\thempfootnote{\arabic{mpfootnote}}
%	\renewcommand{\thempfootnote}{%
%		\setcounter{indianmpfoot}{0}%
%		\addtocounter{indianmpfoot}{\value{mpfootnote}}%
%		\arabic{indianmpfoot}}%
}
%%%%%%%%%%%%%%%%%%%%%%%%%%%%%%%%%%%%%%%%%%%%%%% enddefinition

%%%%%%%%%%%%%%%%%%%%%%%%%%%%%%%%%%%%%%%%%%%%%%%
%%%%%%%%%%%%%%%%%%%%%%%%%%%%%%%%%%%%%%%%%%%%%%%
%%%%%%%%%%%%%%%%%%%%%%%%%%%%%%%%%%%%%%%%%%%%%%%
% Allows the highlighting of lines in lstlisting environments using the key: 
%     linebackgroundcolor = {\btLstHL{line ranges}}
%%%%%%%%%%%%%%%%%%%%%%%%%%%%%%%%%%%%%%%%%%%%%%% begindefinition
\usepackage{listings}

% Define backgroundcolor
    \usepackage[
        style=1,
        skipbelow=\topskip,
        skipabove=\topskip
    ]{mdframed}

    \definecolor{bggray}{rgb}{0.85, 0.85, 0.85}
    \mdfsetup{
        leftmargin = 20pt,
        rightmargin = 20pt,
        backgroundcolor = bggray,
        middlelinecolor = black,
        roundcorner = 15
    }
    \BeforeBeginEnvironment{lstlisting}{\begin{mdframed}\vskip-.5\baselineskip}
    \AfterEndEnvironment{lstlisting}{\end{mdframed}}

\makeatletter
%
% \btIfInRange{number}{range list}{TRUE}{FALSE}
%
% Test if int number <number> is element of a (comma separated) list of ranges
% (such as: {1,3-5,7,10-12,14}) and processes <TRUE> or <FALSE> respectively
%
        \newcount\bt@rangea
        \newcount\bt@rangeb

        \newcommand\btIfInRange[2]{%
            \global\let\bt@inrange\@secondoftwo%
            \edef\bt@rangelist{#2}%
            \foreach \range in \bt@rangelist {%
                \afterassignment\bt@getrangeb%
                \bt@rangea=0\range\relax%
                \pgfmathtruncatemacro\result{ ( #1 >= \bt@rangea) && (#1 <= \bt@rangeb) }%
                \ifnum\result=1\relax%
                    \breakforeach%
                    \global\let\bt@inrange\@firstoftwo%
                \fi%
            }%
            \bt@inrange%
        }

        \newcommand\bt@getrangeb{%
            \@ifnextchar\relax%
            {\bt@rangeb=\bt@rangea}%
            {\@getrangeb}%
        }

        \def\@getrangeb-#1\relax{%
            \ifx\relax#1\relax%
                \bt@rangeb=100000%   \maxdimen is too large for pgfmath
            \else%
                \bt@rangeb=#1\relax%
            \fi%
        }

%
% \btLstHL{range list}
%
    \definecolor{lsthighlight}{RGB}{217, 216, 255}
        \newcommand{\btLstHL}[1]{%
            \btIfInRange{\value{lstnumber}}{#1}%
            {\color{lsthighlight}}%
            {\def\lst@linebgrd}%
        }%

%
% \btInputEmph[listing options]{range list}{file name}
%
        \newcommand{\btLstInputEmph}[3][\empty]{%
            \lstset{%
                linebackgroundcolor=\btLstHL{#2}%
                \lstinputlisting{#3}%
            }% \only
        }

% Patch line number key to call line background macro
        \lst@Key{numbers}{none}{%
            \def\lst@PlaceNumber{\lst@linebgrd}%
            \lstKV@SwitchCases{#1}{%
                none&\\%
                left&\def\lst@PlaceNumber{\llap{\normalfont
                \lst@numberstyle{\thelstnumber}\kern\lst@numbersep}\lst@linebgrd}\\%
                right&\def\lst@PlaceNumber{\rlap{\normalfont
                \kern\linewidth \kern\lst@numbersep
                \lst@numberstyle{\thelstnumber}}\lst@linebgrd}%
            }{%
                \PackageError{Listings}{Numbers #1 unknown}\@ehc%
            }%
        }

% New keys
        \lst@Key{linebackgroundcolor}{}{%
            \def\lst@linebgrdcolor{#1}%
        }
        \lst@Key{linebackgroundsep}{0pt}{%
            \def\lst@linebgrdsep{#1}%
        }
        \lst@Key{linebackgroundwidth}{\linewidth}{%
            \def\lst@linebgrdwidth{#1}%
        }
        \lst@Key{linebackgroundheight}{\ht\strutbox}{%
            \def\lst@linebgrdheight{#1}%
        }
        \lst@Key{linebackgrounddepth}{\dp\strutbox}{%
            \def\lst@linebgrddepth{#1}%
        }
        \lst@Key{linebackgroundcmd}{\color@block}{%
            \def\lst@linebgrdcmd{#1}%
        }

% Line Background macro
        \newcommand{\lst@linebgrd}{%
            \ifx\lst@linebgrdcolor\empty\else
                \rlap{%
                    \lst@basicstyle
                    \color{-.}% By default use the opposite (`-`) of the current color (`.`) as background
                    \lst@linebgrdcolor{%
                        \kern-\dimexpr\lst@linebgrdsep\relax%
                        \lst@linebgrdcmd{\lst@linebgrdwidth}{\lst@linebgrdheight}{\lst@linebgrddepth}%
                    }%
                }%
            \fi
        }

\makeatother
%%%%%%%%%%%%%%%%%%%%%%%%%%%%%%%%%%%%%%%%%%%%%%% enddefinition

%%%%%%%%%%%%%%%%%%%%%%%%%%%%%%%%%%%%%%%%%%%%%%%
%%%%%%%%%%%%%%%%%%%%%%%%%%%%%%%%%%%%%%%%%%%%%%%
%%%%%%%%%%%%%%%%%%%%%%%%%%%%%%%%%%%%%%%%%%%%%%%
% Defines a key=value switch for lstlistings with matchrangestart=<true/false> that allows
% the numbering following the linerange key=value settings:
%%%%%%%%%%%%%%%%%%%%%%%%%%%%%%%%%%%%%%%%%%%%%%% begindefinition
\usepackage{listings}

\makeatletter
\lst@Key{matchrangestart}{false}{\lstKV@SetIf{#1}\lst@ifmatchrangestart}
\def\lst@SkipToFirst{%
    \lst@ifmatchrangestart\c@lstnumber=\numexpr-1+\lst@firstline\fi
    \ifnum \lst@lineno<\lst@firstline
        \def\lst@next{\lst@BeginDropInput\lst@Pmode
        \lst@Let{13}\lst@MSkipToFirst
        \lst@Let{10}\lst@MSkipToFirst}%
        \expandafter\lst@next
    \else
        \expandafter\lst@BOLGobble
    \fi}
\makeatother
%%%%%%%%%%%%%%%%%%%%%%%%%%%%%%%%%%%%%%%%%%%%%%% enddefinition

%%%%%%%%%%%%%%%%%%%%%%%%%%%%%%%%%%%%%%%%%%%%%%%
%%%%%%%%%%%%%%%%%%%%%%%%%%%%%%%%%%%%%%%%%%%%%%%
%%%%%%%%%%%%%%%%%%%%%%%%%%%%%%%%%%%%%%%%%%%%%%%
% Allows the writing of dates specified as \specificdate{YYYY}{MM}{DD}:
%%%%%%%%%%%%%%%%%%%%%%%%%%%%%%%%%%%%%%%%%%%%%%% begindefinition
\newcommand{\specificdate}[3]{\setdate{#1}{#2}{#3} \datedate}
%%%%%%%%%%%%%%%%%%%%%%%%%%%%%%%%%%%%%%%%%%%%%%% enddefinition

%%%%%%%%%%%%%%%%%%%%%%%%%%%%%%%%%%%%%%%%%%%%%%%
%%%%%%%%%%%%%%%%%%%%%%%%%%%%%%%%%%%%%%%%%%%%%%%
%%%%%%%%%%%%%%%%%%%%%%%%%%%%%%%%%%%%%%%%%%%%%%%
% Allows inline comments using \ignore{comment text}:
%%%%%%%%%%%%%%%%%%%%%%%%%%%%%%%%%%%%%%%%%%%%%%% begindefinition
\newcommand{\ignore}[1]{}
%%%%%%%%%%%%%%%%%%%%%%%%%%%%%%%%%%%%%%%%%%%%%%% enddefinition

\settoggle{bwmode}{false}				% enable/disable grayscale mode for the output (see ./tex/command_definitions.tex)

% Settings regarding the used fonts and typographical details:
\usepackage[utf8]{inputenc}			% character encoding of this .tex-file
\usepackage[T1]{fontenc}				% character encoding within the compiled document
\usepackage{libertine\ignore{, libertinust1math}}	% used font/fonts
\usepackage{microtype}				% enables microtypography (can be further configured but the default mode is well)

% Settings regarding the text alignment:
\usepackage
		[
			newcommands,			% the commands \centering, \raggedleft, and \raggedright are redefined to work like \Centering, \RaggedLeft, and \RaggedRight
			newparameters			% the commands \Centering, \RaggedLeft, and \RaggedRight don't behave like vanilla \centering, \raggedleft, and \raggedright 
		]{ragged2e}				% offers new environments for ragged ( and justified) text alignment with many parameters that can be changed using \setlength 

% Settings regarding the used languages:
\usepackage[main = english, ngerman]{babel}	% define the used languages (has influence on different things like hyphenation, date format and figure labels)
\babeltags{eng = english, de = ngerman}		% allows to switch between the loaded languages by using the tags like \begin{<tag>} ... \end{<tag>} or \text<tag>{ ... }
\usepackage[
			detect-all]					% use the font settings of the surrounding text for numbers and units set with siunitx
		{siunitx}						% allows the easy typesetting of numbers, units and combinations including lists and ranges
\sisetup{
	range-phrase = \text{--},				% select the phrase between upper and lower bounds of ranges (here: -)
	binary-units = true,					% enable/disable loading of binary prefixes
	per-mode = fraction,					% select how to display \per (symbol: uses exponents; fraction: uses fractions)
	fraction-function = \sfrac}				% select the kind of fraction used in siunitx (e.g. frac, cfrac, rfrac, sfrac, ...)
\DeclareSIUnit{\euro}{\text{\texteuro}}		% define the unit \euro using the €-symbol
\DeclareSIUnit{\usdollar}{\text{US-\textdollar}}	% define the unit \usdollar using the US-$-symbol
\DeclareSIUnit{\cy}{\text{Cyc.}}				% define the unit \cy used for Cycles using the abbreviation Cyc.
\DeclareSIUnit\century{\text{century}}		% define the unit \century
\DeclareSIUnit\year{\text{year}}				% define the unit \year
\DeclareSIUnit\queries{\text{queries}}		% define the unit \queries
\DeclareSIUnit\transactions{\text{transactions}}	% define the unit \transactions

\usepackage{graphicx}				% allows including of graphics and the scaling and rotating of elements
\usepackage{etoolbox}				% toolbox used by packages and classes
\usepackage{xpatch}					% extends the patching facility of etoolbox
%\usepackage{csquotes}				% context sensitive quotation environment e.g. used by biblatex
%\usepackage{datenumber}				% allows to create a number from a date and especially it allows to create a specific date
%\usepackage{tabularx}				% a tabular* environment that can control the width of columns
\usepackage{multirow}				% allows tabular cells spanning multiple rows
\usepackage{prelim2e}				% marks every page as being a preliminary version when this document is compiled as a draft

\usepackage
		[
			style = alphabetic, 		% select the style of the citation and of the bibliography
			backend = biber		% select the backend that processes the .bib file (run the selected backend instead of BibTeX)
		]{biblatex}					% used to create the bibliography, more modern alternative to the standard BibTeX
\addbibresource[datatype = bibtex]{./tex/references.bib}		% loads the specified bibliography file (in BibTeX format) into BibLaTeX
\usepackage{breakcites}				% multiple citations within one \cite break at the end of the line
\usepackage{hyperref}				% allows hyperlinks within the output document (hyperfootnotes = false to make it compatible with package footmisc)
\usepackage{nameref}				% allows the usage of the command \nameref which prints the title of the referenced label instead of its number

\usepackage{tikz}					% extremely powerful facility to create diagrams
\usetikzlibrary{shapes.geometric, shapes.misc, shapes.callouts, shapes.multipart, shapes.symbols}			% provide several shapes besides the standard ones
\usetikzlibrary{decorations.pathreplacing}	% allows decorated paths without having the original (undecorated) line
\usetikzlibrary{patterns}				% allows the usage of several patterns to fill shapes
\usetikzlibrary{positioning}				% defines additional options for placing nodes conventionally
\usetikzlibrary{calc}					% allows extended coordinate calculation
\usetikzlibrary{spy}					% allows the magnification of parts of a tikz diagram
\usetikzlibrary{chains}				% allows the creation of chains of nodes
\usepackage{pgfplots}				% allows the creation of plots to visualize data
\usepackage{pgfplotstable}			% allows the loading of .csv-files for pgfplots
\usepgflibrary{plotmarks}				% extends the available plot marks used e.g. for pgfplots
\usepgfplotslibrary{fillbetween}			% allows the filling of areas between curves of pgfplots using colors or patterns
\usetikzlibrary{tikzmark}				% allows the definition of coordinates outside of a tikzpicture environment
\usepackage{ifthen}					% allows the usage of the \ifthenelse control structure and some boolean operations with it

\pgfplotsset{compat = 1.14}

\usepackage{xfrac}					% adds the \sfrac fraction mode
\usepackage{rotating}				% allows different kinds of rotations for many kinds of elements
\usepackage{stackengine}				% allows the stacking of elements like symbols
\usepackage{bm}					% adds one way of bold math
\usepackage{ulem}					% allows many kinds of text decorations like underlines or strikes
\robustify\uline						% allows the usage of \uline together with typewriter fonts
\normalem						% \emph uses italic fonts to emphasize text

\usepackage{listings}				% allows the printing of source code
% for a definition of the parameter 'matchrangestart' see ./tex/command_definitions.tex
\usepackage{algpseudocode}			% allows the creation of pseudocode listings

%%%%%%%%%%%%%%%%%%%%%%%%%%%%%%%%%%%%%%%%%%%%%%%%%%%%%%%%%%%%%%%%%%%%%%

\usepackage{url, enumerate, relsize, color, ulem}
\usepackage{anyfontsize}
\usepackage{adjustbox}
\usepackage{listings}

\usepackage{geometry}
\usepackage{setspace}
\usepackage{wrapfig}
\usepackage[para]{footmisc}

\usepackage{scalerel}
\newcommand\dangersign[1][2ex]{%
  \renewcommand\stacktype{L}%
  \scaleto{\stackon[1.3pt]{\color{red}$\triangle$}{\tiny\textrm{!}}}{#1}%
}

\newcommand{\tabitem}{~~\llap{\textbullet}~~}

\defbibenvironment{bibliography}
  {\list{}
     {\settowidth{\labelwidth}{\usebeamertemplate{bibliography item}}%
      \setlength{\leftmargin}{\labelwidth}%
      \setlength{\labelsep}{\biblabelsep}%
      \addtolength{\leftmargin}{\labelsep}%
      \setlength{\itemsep}{\bibitemsep}%
      \setlength{\parsep}{\bibparsep}}}
  {\endlist}
  {\item}
  
\let\thempfootnote\thefootnote

\mode<presentation> {
\usetheme{Dresden}

\newcommand{\frameofframes}{/}
\newcommand{\setframeofframes}[1]{\renewcommand{\frameofframes}{#1}}
\setframeofframes{of}

\setbeamertemplate{headline}
{%
  \begin{beamercolorbox}[colsep=1.5pt]{upper separation line head}
  \end{beamercolorbox}
  \begin{beamercolorbox}{section in head/foot}
    \vskip2pt\insertsectionnavigationhorizontal{\textwidth}{}{}\vskip2pt
  \end{beamercolorbox}%
    \begin{beamercolorbox}[colsep=1.5pt]{middle separation line head}
    \end{beamercolorbox}
    \begin{beamercolorbox}[ht=2.5ex,dp=1.125ex,%
      leftskip=.3cm,rightskip=.3cm plus1fil]{subsection in head/foot}
      \usebeamerfont{subsection in head/foot}\insertsubsectionhead
      \hfill%
      {\usebeamerfont{frame number}\usebeamercolor[fg]{frame number}\insertframenumber~\frameofframes~\inserttotalframenumber}    \end{beamercolorbox}%
  \begin{beamercolorbox}[colsep=1.5pt]{lower separation line head}
  \end{beamercolorbox}
}

%\setbeamertemplate{footline} % To remove the footer line in all slides uncomment this line
%\setbeamertemplate{footline}[page number] % To replace the footer line in all slides with a simple slide count uncomment this line

\setbeamertemplate{navigation symbols}{} % To remove the navigation symbols from the bottom of all slides uncomment this line
}

\usepackage{booktabs} % Allows the use of \toprule, \midrule and \bottomrule in tables

%----------------------------------------------------------------------------------------
%	TITLE PAGE
%----------------------------------------------------------------------------------------

\title[Evaluation of Pointer Swizzling Techniques for DBMS Buffer Management]{Evaluation of Pointer Swizzling Techniques for DBMS Buffer Management} % The short title appears at the bottom of every slide, the full title is only on the title page

\author{Max Gilbert} % Your name
\institute[University of Kaiserslautern] % Your institution as it will appear on the bottom of every slide, may be shorthand to save space
{
University of Kaiserslautern \\ % Your institution for the title page
\medskip
\textit{m\_gilbert13@cs.uni-kl.de} % Your email address
}
\date{\today} % Date, can be changed to a custom date

\newcommand*{\ptsans}{\fontfamily{PTSans-TLF}\selectfont}	% toggle to change to the standard font of the University of Kaiserslautern: PT Sans
\DeclareTextFontCommand{\textptsans}{\ptsans}			% environment to change to the standard font of the University of Kaiserslautern: PT Sans

\iftoggle{bwmode}{
	\definecolor{TUblue}{RGB}{0,0,0}					% The blue color used in the logo of the University of Kaiserslautern is printed black when in grayscale mode
	\definecolor{TUred}{RGB}{127,127,127}					% The red color used in the logo of the University of Kaiserslautern is printed black when in grayscale mode
}{
	\definecolor{TUblue}{RGB}{0,96,142}				% The blue color used in the logo of the University of Kaiserslautern
	\definecolor{TUred}{RGB}{188,38,26}				% The red color used in the logo of the University of Kaiserslautern
}

% The following new commands are tikzpicture-environments containing different logos of the University of Kaiserslautern. They are taken from logos published on their website.
% It defines the following commands: \TULogo, \TULogoWithText, \CSLogo

% The logo of the University of Kaiserslautern ():
% Taken from: http://www.uni-kl.de/fileadmin/prum/tupublic/TU_Logo_ohne_Feld/TUKL_LOGO_4C.svg on the 2016-12-14
% Manipulated using: Inkscape (https://inkscape.org/)
% Converted to TikZ using: svg2tikz (https://github.com/kjellmf/svg2tikz) as an Inkscape extension
\newcommand{\TULogo}[1][1]{
	\begin{tikzpicture}[
		y = 5pt,
		x = 5pt,
		opacity = #1
	]
		% Top part:
		\path[fill = TUblue] (2.898, 23.855) -- (2.898, 20.561) -- (24.299, 20.561)  -- (24.299, 24.034) -- (13.541, 27.197) -- cycle;
		% Left part:
		\path[fill = TUblue] (5.679, 19.307) -- (9.993, 19.307) -- (4.3, 0)  -- (0, 0) -- cycle;
		% Top rectangle:
		\path[fill = TUblue] (17.481, 19.316) rectangle (22.247, 14.727);
		% Middle rectangle:
		\path[fill = TUred] (17.481, 11.953) rectangle (22.247, 7.363);
		% Bottom rectangle:
		\path[fill = TUblue] (17.481, 4.59) rectangle (22.247, 0);
	\end{tikzpicture}
}

% The logo with text of the University of Kaiserslautern:
% Taken from: http://www.uni-kl.de/fileadmin/prum/tupublic/TU_Logo_ohne_Feld/TUKL_LOGO_4C.svg on the 2016-12-14
% Converted to TikZ using: svg2tikz (https://github.com/kjellmf/svg2tikz) as an Inkscape (https://inkscape.org/) extension
\newcommand{\TULogoWithText}{
	\begin{tikzpicture}[
		y = 0.8pt,
		x = 0.8pt,
		yscale = -1,
		xscale = 1
	]
		% KAISERSLAUTERN:
		\path[fill = TUblue] (164.0310,41.3410) -- (164.0310,36.5480) -- (163.8200,35.1050) -- (163.8890,35.1050) -- (164.6070,36.5480) -- (168.0610,41.4050) -- (169.3700,41.4050) -- (169.3700,32.1510) -- (167.6650,32.1510) -- (167.6650,36.9830) -- (167.8760,38.3870) -- (167.8120,38.3870) -- (167.1170,36.9810) -- (163.6440,32.0860) -- (162.3270,32.0860) -- (162.3270,41.3400) -- (164.0310,41.3400) -- cycle(155.4790,33.6880) .. controls (155.5790,33.6630) and (155.7170,33.6450) .. (155.8970,33.6370) .. controls (156.0780,33.6290) and (156.2590,33.6250) .. (156.4440,33.6250) .. controls (156.9200,33.6250) and (157.2800,33.7360) .. (157.5230,33.9570) .. controls (157.7670,34.1800) and (157.8890,34.4880) .. (157.8890,34.8810) .. controls (157.8890,35.4060) and (157.7410,35.7830) .. (157.4420,36.0120) .. controls (157.1450,36.2410) and (156.7460,36.3540) .. (156.2450,36.3540) -- (155.4790,36.3540) -- (155.4790,33.6880) -- cycle(155.4790,41.3410) -- (155.4790,37.5730) -- (156.4560,37.7820) -- (158.5240,41.3410) -- (160.5940,41.3410) -- (158.5110,37.8720) -- (157.8510,37.4330) .. controls (158.3790,37.2490) and (158.8020,36.9190) .. (159.1200,36.4470) .. controls (159.4350,35.9720) and (159.5950,35.3570) .. (159.5950,34.6050) .. controls (159.5950,34.1010) and (159.5020,33.6810) .. (159.3150,33.3430) .. controls (159.1290,33.0070) and (158.8800,32.7410) .. (158.5680,32.5480) .. controls (158.2570,32.3570) and (157.9040,32.2200) .. (157.5090,32.1420) .. controls (157.1140,32.0620) and (156.7140,32.0230) .. (156.3090,32.0230) .. controls (156.1320,32.0230) and (155.9380,32.0290) .. (155.7270,32.0370) .. controls (155.5160,32.0450) and (155.3000,32.0580) .. (155.0780,32.0760) .. controls (154.8540,32.0920) and (154.6310,32.1170) .. (154.4050,32.1480) .. controls (154.1790,32.1810) and (153.9700,32.2140) .. (153.7770,32.2480) -- (153.7770,41.3420) -- (155.4790,41.3420) -- cycle(151.9100,41.3410) -- (151.9100,39.7390) -- (148.1030,39.7390) -- (148.1030,37.4970) -- (151.5180,37.4970) -- (151.5180,35.8930) -- (148.1030,35.8930) -- (148.1030,33.7520) -- (151.8450,33.7520) -- (151.8450,32.1500) -- (146.3980,32.1500) -- (146.3980,41.3390) -- (151.9100,41.3390) -- cycle(140.2020,33.7530) -- (140.2020,41.3410) -- (141.9070,41.3410) -- (141.9070,33.7530) -- (144.6810,33.7530) -- (144.6810,32.1510) -- (137.4200,32.1510) -- (137.4200,33.7530) -- (140.2020,33.7530) -- cycle(132.7710,41.5010) .. controls (133.2500,41.5010) and (133.6930,41.4350) .. (134.0960,41.3040) .. controls (134.5000,41.1710) and (134.8420,40.9640) .. (135.1220,40.6850) .. controls (135.4020,40.4060) and (135.6200,40.0500) .. (135.7770,39.6210) .. controls (135.9320,39.1910) and (136.0100,38.6800) .. (136.0100,38.0860) -- (136.0100,32.1520) -- (134.3050,32.1520) -- (134.3050,37.9590) .. controls (134.3050,38.6390) and (134.1820,39.1310) .. (133.9360,39.4390) .. controls (133.6890,39.7460) and (133.2900,39.9000) .. (132.7390,39.9000) .. controls (132.4610,39.9000) and (132.2170,39.8670) .. (132.0070,39.8000) .. controls (131.7970,39.7330) and (131.6220,39.6240) .. (131.4800,39.4700) .. controls (131.3390,39.3160) and (131.2360,39.1160) .. (131.1690,38.8680) .. controls (131.1020,38.6200) and (131.0700,38.3150) .. (131.0700,37.9580) -- (131.0700,32.1510) -- (129.3650,32.1510) -- (129.3650,38.3090) .. controls (129.3660,40.4370) and (130.5010,41.5010) .. (132.7710,41.5010)(123.9030,35.8310) -- (124.1770,34.3760) -- (124.2410,34.3760) -- (124.5220,35.8170) -- (125.2030,37.8620) -- (123.2280,37.8620) -- (123.9030,35.8310) -- cycle(122.0790,41.3410) -- (122.7000,39.4640) -- (125.7220,39.4640) -- (126.3310,41.3410) -- (128.0350,41.3410) -- (124.7280,32.0870) -- (123.4420,32.0870) -- (120.2810,41.3410) -- (122.0790,41.3410) -- cycle(119.2220,41.3410) -- (119.2220,39.7390) -- (115.1150,39.7390) -- (115.1150,32.1510) -- (113.4100,32.1510) -- (113.4100,41.3400) -- (119.2220,41.3400) -- cycle(106.5570,41.3410) .. controls (107.0430,41.4640) and (107.6040,41.5290) .. (108.2410,41.5290) .. controls (108.7220,41.5290) and (109.1670,41.4700) .. (109.5720,41.3530) .. controls (109.9760,41.2380) and (110.3220,41.0600) .. (110.6060,40.8240) .. controls (110.8910,40.5880) and (111.1130,40.2910) .. (111.2700,39.9330) .. controls (111.4280,39.5760) and (111.5070,39.1480) .. (111.5070,38.6540) .. controls (111.5070,38.1770) and (111.4160,37.7790) .. (111.2330,37.4570) .. controls (111.0510,37.1350) and (110.8230,36.8650) .. (110.5450,36.6460) .. controls (110.2680,36.4290) and (109.9500,36.2380) .. (109.5900,36.0740) .. controls (109.2310,35.9080) and (108.8940,35.7540) .. (108.5810,35.6110) .. controls (108.2680,35.4680) and (107.9950,35.3080) .. (107.7620,35.1360) .. controls (107.5300,34.9620) and (107.4120,34.7470) .. (107.4120,34.4880) .. controls (107.4120,34.2090) and (107.5230,33.9880) .. (107.7470,33.8220) .. controls (107.9690,33.6580) and (108.2910,33.5760) .. (108.7110,33.5760) .. controls (109.1480,33.5760) and (109.5540,33.6250) .. (109.9300,33.7220) .. controls (110.3050,33.8240) and (110.5860,33.9330) .. (110.7740,34.0540) -- (111.3070,32.5250) .. controls (111.0130,32.3450) and (110.6360,32.2090) .. (110.1770,32.1150) .. controls (109.7170,32.0190) and (109.2280,31.9720) .. (108.7110,31.9720) .. controls (108.2630,31.9720) and (107.8570,32.0250) .. (107.4920,32.1300) .. controls (107.1270,32.2370) and (106.8100,32.4000) .. (106.5440,32.6200) .. controls (106.2780,32.8400) and (106.0700,33.1180) .. (105.9250,33.4500) .. controls (105.7800,33.7840) and (105.7080,34.1750) .. (105.7080,34.6240) .. controls (105.7080,35.1340) and (105.8100,35.5540) .. (106.0150,35.8840) .. controls (106.2200,36.2120) and (106.4790,36.4880) .. (106.7910,36.7060) .. controls (107.1030,36.9250) and (107.4380,37.1140) .. (107.7990,37.2740) .. controls (108.1600,37.4340) and (108.4950,37.5830) .. (108.8080,37.7210) .. controls (109.1210,37.8620) and (109.3660,38.0140) .. (109.5400,38.1840) .. controls (109.7140,38.3540) and (109.8030,38.5790) .. (109.8030,38.8540) .. controls (109.8030,39.2150) and (109.6600,39.4830) .. (109.3760,39.6610) .. controls (109.0910,39.8370) and (108.6810,39.9250) .. (108.1450,39.9250) .. controls (107.9260,39.9250) and (107.7140,39.9090) .. (107.5080,39.8740) .. controls (107.3010,39.8410) and (107.1050,39.7980) .. (106.9200,39.7450) .. controls (106.7330,39.6920) and (106.5660,39.6360) .. (106.4190,39.5750) .. controls (106.2700,39.5130) and (106.1480,39.4580) .. (106.0540,39.4070) -- (105.4770,40.9620) .. controls (105.7110,41.0890) and (106.0710,41.2160) .. (106.5570,41.3410)(99.0840,33.6880) .. controls (99.1830,33.6630) and (99.3220,33.6450) .. (99.5020,33.6370) .. controls (99.6820,33.6290) and (99.8640,33.6250) .. (100.0480,33.6250) .. controls (100.5230,33.6250) and (100.8830,33.7360) .. (101.1270,33.9570) .. controls (101.3720,34.1800) and (101.4940,34.4880) .. (101.4940,34.8810) .. controls (101.4940,35.4060) and (101.3450,35.7830) .. (101.0470,36.0120) .. controls (100.7490,36.2410) and (100.3500,36.3540) .. (99.8490,36.3540) -- (99.0840,36.3540) -- (99.0840,33.6880) -- cycle(99.0840,41.3410) -- (99.0840,37.5730) -- (100.0600,37.7820) -- (102.1280,41.3410) -- (104.1980,41.3410) -- (102.1160,37.8720) -- (101.4550,37.4330) .. controls (101.9840,37.2490) and (102.4070,36.9190) .. (102.7230,36.4470) .. controls (103.0400,35.9720) and (103.1990,35.3570) .. (103.1990,34.6050) .. controls (103.1990,34.1010) and (103.1050,33.6810) .. (102.9200,33.3430) .. controls (102.7330,33.0070) and (102.4850,32.7410) .. (102.1730,32.5480) .. controls (101.8610,32.3570) and (101.5080,32.2200) .. (101.1130,32.1420) .. controls (100.7180,32.0620) and (100.3190,32.0230) .. (99.9140,32.0230) .. controls (99.7360,32.0230) and (99.5420,32.0290) .. (99.3320,32.0370) .. controls (99.1210,32.0450) and (98.9040,32.0580) .. (98.6810,32.0760) .. controls (98.4580,32.0920) and (98.2340,32.1170) .. (98.0090,32.1480) .. controls (97.7840,32.1810) and (97.5740,32.2140) .. (97.3800,32.2480) -- (97.3800,41.3420) -- (99.0840,41.3420) -- cycle(95.2520,41.3410) -- (95.2520,39.7390) -- (91.4460,39.7390) -- (91.4460,37.4970) -- (94.8610,37.4970) -- (94.8610,35.8930) -- (91.4460,35.8930) -- (91.4460,33.7520) -- (95.1880,33.7520) -- (95.1880,32.1500) -- (89.7410,32.1500) -- (89.7410,41.3390) -- (95.2520,41.3390) -- cycle(82.6260,41.3410) .. controls (83.1120,41.4640) and (83.6730,41.5290) .. (84.3090,41.5290) .. controls (84.7910,41.5290) and (85.2350,41.4700) .. (85.6400,41.3530) .. controls (86.0450,41.2380) and (86.3900,41.0600) .. (86.6750,40.8240) .. controls (86.9590,40.5880) and (87.1810,40.2910) .. (87.3390,39.9330) .. controls (87.4970,39.5750) and (87.5760,39.1480) .. (87.5760,38.6540) .. controls (87.5760,38.1770) and (87.4840,37.7790) .. (87.3020,37.4570) .. controls (87.1200,37.1350) and (86.8900,36.8650) .. (86.6130,36.6460) .. controls (86.3360,36.4290) and (86.0180,36.2380) .. (85.6580,36.0740) .. controls (85.2990,35.9080) and (84.9620,35.7540) .. (84.6490,35.6110) .. controls (84.3360,35.4680) and (84.0630,35.3080) .. (83.8300,35.1360) .. controls (83.5970,34.9620) and (83.4800,34.7470) .. (83.4800,34.4880) .. controls (83.4800,34.2090) and (83.5920,33.9880) .. (83.8150,33.8220) .. controls (84.0370,33.6580) and (84.3590,33.5760) .. (84.7780,33.5760) .. controls (85.2160,33.5760) and (85.6220,33.6250) .. (85.9980,33.7220) .. controls (86.3730,33.8240) and (86.6540,33.9330) .. (86.8410,34.0540) -- (87.3750,32.5250) .. controls (87.0810,32.3450) and (86.7040,32.2090) .. (86.2450,32.1150) .. controls (85.7850,32.0190) and (85.2960,31.9720) .. (84.7780,31.9720) .. controls (84.3310,31.9720) and (83.9250,32.0250) .. (83.5600,32.1300) .. controls (83.1950,32.2370) and (82.8790,32.4000) .. (82.6120,32.6200) .. controls (82.3440,32.8410) and (82.1380,33.1180) .. (81.9930,33.4500) .. controls (81.8480,33.7840) and (81.7760,34.1750) .. (81.7760,34.6240) .. controls (81.7760,35.1340) and (81.8780,35.5540) .. (82.0830,35.8840) .. controls (82.2880,36.2120) and (82.5470,36.4880) .. (82.8590,36.7060) .. controls (83.1710,36.9240) and (83.5070,37.1140) .. (83.8670,37.2740) .. controls (84.2270,37.4340) and (84.5630,37.5830) .. (84.8760,37.7210) .. controls (85.1890,37.8620) and (85.4330,38.0140) .. (85.6080,38.1840) .. controls (85.7830,38.3540) and (85.8710,38.5790) .. (85.8710,38.8540) .. controls (85.8710,39.2150) and (85.7280,39.4830) .. (85.4430,39.6610) .. controls (85.1590,39.8370) and (84.7480,39.9250) .. (84.2130,39.9250) .. controls (83.9940,39.9250) and (83.7820,39.9090) .. (83.5760,39.8740) .. controls (83.3690,39.8410) and (83.1730,39.7980) .. (82.9880,39.7450) .. controls (82.8020,39.6920) and (82.6350,39.6360) .. (82.4870,39.5750) .. controls (82.3380,39.5130) and (82.2160,39.4580) .. (82.1210,39.4070) -- (81.5450,40.9620) .. controls (81.7800,41.0890) and (82.1400,41.2160) .. (82.6260,41.3410)(79.7610,32.1510) -- (78.0560,32.1510) -- (78.0560,41.3400) -- (79.7610,41.3400) -- (79.7610,32.1510) -- cycle(72.1060,35.8310) -- (72.3820,34.3760) -- (72.4460,34.3760) -- (72.7280,35.8170) -- (73.4070,37.8620) -- (71.4340,37.8620) -- (72.1060,35.8310) -- cycle(70.2820,41.3410) -- (70.9030,39.4640) -- (73.9260,39.4640) -- (74.5350,41.3410) -- (76.2400,41.3410) -- (72.9330,32.0870) -- (71.6470,32.0870) -- (68.4850,41.3410) -- (70.2820,41.3410) -- cycle(62.0580,41.3410) -- (62.0580,37.4560) -- (62.5690,37.4560) -- (65.2560,41.3410) -- (67.4730,41.3410) -- (64.4440,37.0620) -- (63.6760,36.5440) -- (64.3890,36.0480) -- (67.0700,32.1520) -- (65.0190,32.1520) -- (62.4820,36.0430) -- (62.0580,36.2210) -- (62.0580,32.1530) -- (60.3530,32.1530) -- (60.3530,41.3420) -- (62.0580,41.3420) -- cycle;
		% TECHNISCHE UNIVERSITÄT:
		\path[fill=TUred] (166.8350,23.1740) -- (166.8350,28.6690) -- (168.0690,28.6690) -- (168.0690,23.1740) -- (170.0790,23.1740) -- (170.0790,22.0140) -- (164.8210,22.0140) -- (164.8210,23.1740) -- (166.8350,23.1740) -- cycle(162.7020,21.4690) .. controls (162.8160,21.5720) and (162.9970,21.6240) .. (163.2450,21.6240) .. controls (163.4980,21.6240) and (163.6830,21.5720) .. (163.7950,21.4690) .. controls (163.9080,21.3650) and (163.9640,21.2260) .. (163.9640,21.0510) .. controls (163.9640,20.8740) and (163.9080,20.7310) .. (163.7950,20.6240) .. controls (163.6830,20.5170) and (163.4980,20.4640) .. (163.2450,20.4640) .. controls (162.9970,20.4640) and (162.8160,20.5180) .. (162.7020,20.6270) .. controls (162.5870,20.7350) and (162.5300,20.8770) .. (162.5300,21.0510) .. controls (162.5300,21.2260) and (162.5870,21.3650) .. (162.7020,21.4690)(160.6690,21.4690) .. controls (160.7840,21.5720) and (160.9680,21.6240) .. (161.2210,21.6240) .. controls (161.4660,21.6240) and (161.6460,21.5720) .. (161.7600,21.4690) .. controls (161.8740,21.3650) and (161.9320,21.2260) .. (161.9320,21.0510) .. controls (161.9320,20.8740) and (161.8750,20.7310) .. (161.7620,20.6240) .. controls (161.6500,20.5170) and (161.4690,20.4640) .. (161.2210,20.4640) .. controls (160.9680,20.4640) and (160.7840,20.5180) .. (160.6690,20.6270) .. controls (160.5560,20.7350) and (160.4970,20.8770) .. (160.4970,21.0510) .. controls (160.4970,21.2260) and (160.5560,21.3650) .. (160.6690,21.4690)(161.9410,24.6780) -- (162.1400,23.6240) -- (162.1870,23.6240) -- (162.3910,24.6680) -- (162.8820,26.1490) -- (161.4530,26.1490) -- (161.9410,24.6780) -- cycle(160.6200,28.6690) -- (161.0700,27.3090) -- (163.2580,27.3090) -- (163.6990,28.6690) -- (164.9330,28.6690) -- (162.5380,21.9670) -- (161.6060,21.9670) -- (159.3170,28.6690) -- (160.6200,28.6690) -- cycle(156.2580,23.1740) -- (156.2580,28.6690) -- (157.4920,28.6690) -- (157.4920,23.1740) -- (159.5020,23.1740) -- (159.5020,22.0140) -- (154.2440,22.0140) -- (154.2440,23.1740) -- (156.2580,23.1740) -- cycle(153.3940,22.0140) -- (152.1600,22.0140) -- (152.1600,28.6690) -- (153.3940,28.6690) -- (153.3940,22.0140) -- cycle(147.3960,28.6680) .. controls (147.7480,28.7580) and (148.1540,28.8030) .. (148.6150,28.8030) .. controls (148.9640,28.8030) and (149.2850,28.7610) .. (149.5790,28.6770) .. controls (149.8720,28.5930) and (150.1220,28.4650) .. (150.3280,28.2940) .. controls (150.5340,28.1220) and (150.6940,27.9070) .. (150.8080,27.6480) .. controls (150.9220,27.3900) and (150.9800,27.0810) .. (150.9800,26.7220) .. controls (150.9800,26.3770) and (150.9140,26.0880) .. (150.7820,25.8540) .. controls (150.6500,25.6210) and (150.4840,25.4260) .. (150.2830,25.2680) .. controls (150.0830,25.1110) and (149.8520,24.9720) .. (149.5920,24.8530) .. controls (149.3310,24.7340) and (149.0880,24.6220) .. (148.8620,24.5190) .. controls (148.6340,24.4150) and (148.4370,24.3000) .. (148.2690,24.1740) .. controls (148.1000,24.0490) and (148.0160,23.8930) .. (148.0160,23.7060) .. controls (148.0160,23.5040) and (148.0950,23.3430) .. (148.2570,23.2230) .. controls (148.4180,23.1040) and (148.6520,23.0440) .. (148.9550,23.0440) .. controls (149.2710,23.0440) and (149.5660,23.0800) .. (149.8370,23.1520) .. controls (150.1090,23.2240) and (150.3140,23.3040) .. (150.4480,23.3920) -- (150.8350,22.2850) .. controls (150.6220,22.1540) and (150.3500,22.0550) .. (150.0170,21.9870) .. controls (149.6840,21.9180) and (149.3290,21.8840) .. (148.9550,21.8840) .. controls (148.6300,21.8840) and (148.3360,21.9220) .. (148.0720,21.9990) .. controls (147.8080,22.0760) and (147.5790,22.1940) .. (147.3850,22.3530) .. controls (147.1920,22.5130) and (147.0420,22.7130) .. (146.9380,22.9550) .. controls (146.8330,23.1960) and (146.7800,23.4790) .. (146.7800,23.8040) .. controls (146.7800,24.1740) and (146.8540,24.4780) .. (147.0030,24.7160) .. controls (147.1510,24.9550) and (147.3390,25.1530) .. (147.5650,25.3120) .. controls (147.7910,25.4710) and (148.0350,25.6080) .. (148.2940,25.7240) .. controls (148.5560,25.8390) and (148.7990,25.9480) .. (149.0250,26.0480) .. controls (149.2520,26.1490) and (149.4280,26.2610) .. (149.5550,26.3840) .. controls (149.6820,26.5070) and (149.7450,26.6680) .. (149.7450,26.8690) .. controls (149.7450,27.1300) and (149.6410,27.3240) .. (149.4350,27.4520) .. controls (149.2290,27.5790) and (148.9320,27.6430) .. (148.5430,27.6430) .. controls (148.3850,27.6430) and (148.2310,27.6310) .. (148.0820,27.6060) .. controls (147.9330,27.5820) and (147.7900,27.5510) .. (147.6560,27.5130) .. controls (147.5210,27.4740) and (147.4000,27.4330) .. (147.2930,27.3890) .. controls (147.1860,27.3450) and (147.0980,27.3040) .. (147.0290,27.2670) -- (146.6110,28.3940) .. controls (146.7820,28.4860) and (147.0430,28.5770) .. (147.3960,28.6680)(142.4110,23.1280) .. controls (142.4820,23.1090) and (142.5830,23.0970) .. (142.7140,23.0900) .. controls (142.8440,23.0840) and (142.9770,23.0810) .. (143.1100,23.0810) .. controls (143.4540,23.0810) and (143.7140,23.1620) .. (143.8920,23.3220) .. controls (144.0690,23.4830) and (144.1570,23.7060) .. (144.1570,23.9910) .. controls (144.1570,24.3710) and (144.0490,24.6440) .. (143.8340,24.8100) .. controls (143.6180,24.9750) and (143.3290,25.0580) .. (142.9660,25.0580) -- (142.4120,25.0580) -- (142.4120,23.1280) -- cycle(142.4110,28.6690) -- (142.4110,25.9400) -- (143.1180,26.0910) -- (144.6150,28.6690) -- (146.1140,28.6690) -- (144.6060,26.1570) -- (144.1270,25.8380) .. controls (144.5100,25.7050) and (144.8160,25.4670) .. (145.0450,25.1230) .. controls (145.2750,24.7800) and (145.3900,24.3360) .. (145.3900,23.7910) .. controls (145.3900,23.4260) and (145.3230,23.1210) .. (145.1880,22.8770) .. controls (145.0520,22.6330) and (144.8730,22.4410) .. (144.6470,22.3020) .. controls (144.4200,22.1620) and (144.1660,22.0640) .. (143.8790,22.0070) .. controls (143.5920,21.9500) and (143.3040,21.9210) .. (143.0100,21.9210) .. controls (142.8820,21.9210) and (142.7410,21.9240) .. (142.5890,21.9300) .. controls (142.4370,21.9370) and (142.2790,21.9460) .. (142.1180,21.9580) .. controls (141.9560,21.9710) and (141.7940,21.9880) .. (141.6310,22.0110) .. controls (141.4680,22.0350) and (141.3170,22.0590) .. (141.1760,22.0830) -- (141.1760,28.6690) -- (142.4110,28.6690) -- cycle(140.1590,28.6690) -- (140.1590,27.5080) -- (137.4030,27.5080) -- (137.4030,25.8840) -- (139.8760,25.8840) -- (139.8760,24.7240) -- (137.4030,24.7240) -- (137.4030,23.1740) -- (140.1130,23.1740) -- (140.1130,22.0140) -- (136.1680,22.0140) -- (136.1680,28.6690) -- (140.1590,28.6690) -- cycle(132.1210,28.7150) -- (133.0490,28.7150) -- (135.5030,22.0140) -- (134.2700,22.0140) -- (132.9980,25.9120) -- (132.8090,27.0540) -- (132.7620,27.0540) -- (132.5900,25.9210) -- (131.2550,22.0140) -- (129.7500,22.0140) -- (132.1210,28.7150) -- cycle(129.0200,22.0140) -- (127.7860,22.0140) -- (127.7860,28.6690) -- (129.0200,28.6690) -- (129.0200,22.0140) -- cycle(122.5950,28.6690) -- (122.5950,25.1970) -- (122.4430,24.1530) -- (122.4940,24.1530) -- (123.0150,25.1970) -- (125.5160,28.7150) -- (126.4620,28.7150) -- (126.4620,22.0140) -- (125.2280,22.0140) -- (125.2280,25.5130) -- (125.3800,26.5290) -- (125.3340,26.5290) -- (124.8300,25.5120) -- (122.3160,21.9670) -- (121.3620,21.9670) -- (121.3620,28.6690) -- (122.5950,28.6690) -- cycle(117.7930,28.7860) .. controls (118.1410,28.7860) and (118.4600,28.7370) .. (118.7530,28.6410) .. controls (119.0450,28.5450) and (119.2930,28.3960) .. (119.4960,28.1930) .. controls (119.6990,27.9900) and (119.8570,27.7340) .. (119.9700,27.4230) .. controls (120.0830,27.1120) and (120.1400,26.7410) .. (120.1400,26.3110) -- (120.1400,22.0140) -- (118.9060,22.0140) -- (118.9060,26.2180) .. controls (118.9060,26.7100) and (118.8160,27.0680) .. (118.6370,27.2900) .. controls (118.4590,27.5130) and (118.1700,27.6250) .. (117.7710,27.6250) .. controls (117.5700,27.6250) and (117.3930,27.6010) .. (117.2410,27.5530) .. controls (117.0890,27.5050) and (116.9620,27.4250) .. (116.8600,27.3140) .. controls (116.7560,27.2020) and (116.6820,27.0570) .. (116.6330,26.8770) .. controls (116.5850,26.6980) and (116.5620,26.4780) .. (116.5620,26.2180) -- (116.5620,22.0140) -- (115.3280,22.0140) -- (115.3280,26.4740) .. controls (115.3270,28.0140) and (116.1490,28.7860) .. (117.7930,28.7860)(111.8860,28.6690) -- (111.8860,27.5080) -- (109.1290,27.5080) -- (109.1290,25.8840) -- (111.6030,25.8840) -- (111.6030,24.7240) -- (109.1290,24.7240) -- (109.1290,23.1740) -- (111.8390,23.1740) -- (111.8390,22.0140) -- (107.8960,22.0140) -- (107.8960,28.6690) -- (111.8860,28.6690) -- cycle(102.8960,28.6690) -- (102.8960,25.8840) -- (105.3360,25.8840) -- (105.3360,28.6690) -- (106.5710,28.6690) -- (106.5710,22.0140) -- (105.3360,22.0140) -- (105.3360,24.7240) -- (102.8960,24.7240) -- (102.8960,22.0140) -- (101.6620,22.0140) -- (101.6620,28.6690) -- (102.8960,28.6690) -- cycle(96.3130,26.9470) .. controls (96.4580,27.3870) and (96.6530,27.7450) .. (96.8990,28.0200) .. controls (97.1440,28.2950) and (97.4450,28.4940) .. (97.8020,28.6180) .. controls (98.1590,28.7420) and (98.5360,28.8030) .. (98.9360,28.8030) .. controls (99.2630,28.8030) and (99.5850,28.7720) .. (99.8990,28.7090) .. controls (100.2120,28.6460) and (100.4720,28.5420) .. (100.6750,28.3980) -- (100.4060,27.3360) .. controls (100.2600,27.4260) and (100.0890,27.5000) .. (99.8930,27.5570) .. controls (99.6960,27.6140) and (99.4560,27.6430) .. (99.1710,27.6430) .. controls (98.8680,27.6430) and (98.6000,27.5870) .. (98.3680,27.4760) .. controls (98.1360,27.3640) and (97.9430,27.2080) .. (97.7880,27.0080) .. controls (97.6340,26.8080) and (97.5180,26.5670) .. (97.4430,26.2850) .. controls (97.3670,26.0030) and (97.3290,25.6900) .. (97.3290,25.3460) .. controls (97.3290,24.5580) and (97.4890,23.9780) .. (97.8090,23.6040) .. controls (98.1290,23.2310) and (98.5520,23.0440) .. (99.0780,23.0440) .. controls (99.3630,23.0440) and (99.6050,23.0600) .. (99.8050,23.0910) .. controls (100.0040,23.1220) and (100.1770,23.1730) .. (100.3220,23.2440) -- (100.5770,22.1390) .. controls (100.4070,22.0680) and (100.1910,22.0080) .. (99.9280,21.9580) .. controls (99.6650,21.9090) and (99.3430,21.8840) .. (98.9620,21.8840) .. controls (98.6070,21.8840) and (98.2520,21.9430) .. (97.8970,22.0610) .. controls (97.5430,22.1780) and (97.2360,22.3720) .. (96.9750,22.6410) .. controls (96.7140,22.9110) and (96.5020,23.2660) .. (96.3390,23.7060) .. controls (96.1760,24.1450) and (96.0940,24.6920) .. (96.0940,25.3460) .. controls (96.0950,25.9730) and (96.1680,26.5070) .. (96.3130,26.9470)(91.6750,28.6680) .. controls (92.0270,28.7580) and (92.4330,28.8030) .. (92.8940,28.8030) .. controls (93.2430,28.8030) and (93.5640,28.7610) .. (93.8570,28.6770) .. controls (94.1510,28.5930) and (94.4010,28.4650) .. (94.6070,28.2940) .. controls (94.8130,28.1220) and (94.9730,27.9070) .. (95.0870,27.6480) .. controls (95.2020,27.3900) and (95.2590,27.0810) .. (95.2590,26.7220) .. controls (95.2590,26.3770) and (95.1930,26.0880) .. (95.0610,25.8540) .. controls (94.9290,25.6210) and (94.7630,25.4260) .. (94.5620,25.2680) .. controls (94.3620,25.1110) and (94.1310,24.9720) .. (93.8710,24.8530) .. controls (93.6100,24.7340) and (93.3660,24.6220) .. (93.1400,24.5190) .. controls (92.9130,24.4150) and (92.7150,24.3000) .. (92.5470,24.1740) .. controls (92.3780,24.0490) and (92.2940,23.8930) .. (92.2940,23.7060) .. controls (92.2940,23.5040) and (92.3740,23.3430) .. (92.5360,23.2230) .. controls (92.6970,23.1040) and (92.9290,23.0440) .. (93.2340,23.0440) .. controls (93.5500,23.0440) and (93.8450,23.0800) .. (94.1160,23.1520) .. controls (94.3880,23.2240) and (94.5920,23.3040) .. (94.7270,23.3920) -- (95.1140,22.2850) .. controls (94.9010,22.1540) and (94.6280,22.0550) .. (94.2950,21.9870) .. controls (93.9620,21.9180) and (93.6080,21.8840) .. (93.2340,21.8840) .. controls (92.9090,21.8840) and (92.6150,21.9220) .. (92.3510,21.9990) .. controls (92.0870,22.0760) and (91.8580,22.1940) .. (91.6650,22.3530) .. controls (91.4710,22.5130) and (91.3220,22.7130) .. (91.2170,22.9550) .. controls (91.1120,23.1960) and (91.0590,23.4790) .. (91.0590,23.8040) .. controls (91.0590,24.1740) and (91.1330,24.4780) .. (91.2820,24.7160) .. controls (91.4300,24.9550) and (91.6180,25.1530) .. (91.8430,25.3120) .. controls (92.0690,25.4710) and (92.3120,25.6080) .. (92.5730,25.7240) .. controls (92.8340,25.8390) and (93.0780,25.9480) .. (93.3040,26.0480) .. controls (93.5310,26.1490) and (93.7080,26.2610) .. (93.8340,26.3840) .. controls (93.9610,26.5070) and (94.0250,26.6680) .. (94.0250,26.8690) .. controls (94.0250,27.1300) and (93.9210,27.3240) .. (93.7150,27.4520) .. controls (93.5090,27.5790) and (93.2120,27.6430) .. (92.8240,27.6430) .. controls (92.6660,27.6430) and (92.5120,27.6310) .. (92.3620,27.6060) .. controls (92.2130,27.5820) and (92.0710,27.5510) .. (91.9370,27.5130) .. controls (91.8020,27.4740) and (91.6810,27.4330) .. (91.5740,27.3890) .. controls (91.4660,27.3450) and (91.3780,27.3040) .. (91.3090,27.2670) -- (90.8920,28.3940) .. controls (91.0620,28.4860) and (91.3230,28.5770) .. (91.6750,28.6680)(89.8370,22.0140) -- (88.6030,22.0140) -- (88.6030,28.6690) -- (89.8370,28.6690) -- (89.8370,22.0140) -- cycle(83.4140,28.6690) -- (83.4140,25.1970) -- (83.2600,24.1530) -- (83.3110,24.1530) -- (83.8310,25.1970) -- (86.3330,28.7150) -- (87.2790,28.7150) -- (87.2790,22.0140) -- (86.0450,22.0140) -- (86.0450,25.5130) -- (86.1970,26.5290) -- (86.1510,26.5290) -- (85.6470,25.5120) -- (83.1330,21.9670) -- (82.1790,21.9670) -- (82.1790,28.6690) -- (83.4140,28.6690) -- cycle(77.1800,28.6690) -- (77.1800,25.8840) -- (79.6210,25.8840) -- (79.6210,28.6690) -- (80.8550,28.6690) -- (80.8550,22.0140) -- (79.6210,22.0140) -- (79.6210,24.7240) -- (77.1800,24.7240) -- (77.1800,22.0140) -- (75.9450,22.0140) -- (75.9450,28.6690) -- (77.1800,28.6690) -- cycle(70.5980,26.9470) .. controls (70.7430,27.3870) and (70.9380,27.7450) .. (71.1830,28.0200) .. controls (71.4290,28.2950) and (71.7300,28.4940) .. (72.0870,28.6180) .. controls (72.4430,28.7420) and (72.8210,28.8030) .. (73.2200,28.8030) .. controls (73.5480,28.8030) and (73.8690,28.7720) .. (74.1830,28.7090) .. controls (74.4970,28.6460) and (74.7560,28.5420) .. (74.9600,28.3980) -- (74.6910,27.3360) .. controls (74.5460,27.4260) and (74.3750,27.5000) .. (74.1780,27.5570) .. controls (73.9820,27.6140) and (73.7410,27.6430) .. (73.4570,27.6430) .. controls (73.1540,27.6430) and (72.8860,27.5870) .. (72.6540,27.4760) .. controls (72.4220,27.3640) and (72.2290,27.2080) .. (72.0740,27.0080) .. controls (71.9190,26.8080) and (71.8040,26.5670) .. (71.7280,26.2850) .. controls (71.6520,26.0030) and (71.6140,25.6900) .. (71.6140,25.3460) .. controls (71.6140,24.5580) and (71.7750,23.9780) .. (72.0950,23.6040) .. controls (72.4150,23.2310) and (72.8380,23.0440) .. (73.3640,23.0440) .. controls (73.6490,23.0440) and (73.8910,23.0600) .. (74.0900,23.0910) .. controls (74.2900,23.1220) and (74.4620,23.1730) .. (74.6080,23.2440) -- (74.8630,22.1390) .. controls (74.6930,22.0680) and (74.4760,22.0080) .. (74.2130,21.9580) .. controls (73.9500,21.9090) and (73.6290,21.8840) .. (73.2480,21.8840) .. controls (72.8920,21.8840) and (72.5370,21.9430) .. (72.1830,22.0610) .. controls (71.8290,22.1780) and (71.5210,22.3720) .. (71.2610,22.6410) .. controls (71.0000,22.9110) and (70.7880,23.2660) .. (70.6250,23.7060) .. controls (70.4620,24.1450) and (70.3800,24.6920) .. (70.3800,25.3460) .. controls (70.3800,25.9730) and (70.4530,26.5070) .. (70.5980,26.9470)(69.5710,28.6690) -- (69.5710,27.5080) -- (66.8150,27.5080) -- (66.8150,25.8840) -- (69.2880,25.8840) -- (69.2880,24.7240) -- (66.8150,24.7240) -- (66.8150,23.1740) -- (69.5250,23.1740) -- (69.5250,22.0140) -- (65.5800,22.0140) -- (65.5800,28.6690) -- (69.5710,28.6690) -- cycle(61.6160,23.1740) -- (61.6160,28.6690) -- (62.8500,28.6690) -- (62.8500,23.1740) -- (64.8600,23.1740) -- (64.8600,22.0140) -- (59.6020,22.0140) -- (59.6020,23.1740) -- (61.6160,23.1740) -- cycle;

		% Top part:
		\path[fill = TUblue] (31.2450,17.5150) -- (31.2450,20.8090) -- (52.6440,20.8090) -- (52.6460,17.3360) -- (41.8880,14.1730) -- cycle;
		% Left part:
		\path[fill = TUblue] (34.0260,22.0630) -- (38.3400,22.0630) -- (32.6470,41.3700) -- (28.3470,41.3700) -- cycle;
		% Top rectangle:
		\path[fill = TUblue,rounded corners=0.0000cm] (45.8280,22.0540) rectangle (50.5940,26.6430);
		% Middle rectangle:
		\path[fill = TUred,rounded corners=0.0000cm] (45.8280,29.4170) rectangle (50.5940,34.0070);
		% Bottom rectangle:
		\path[fill = TUblue,rounded corners=0.0000cm] (45.8280,36.7800) rectangle (50.5940,41.3700);
	\end{tikzpicture}
}

% Colors needed for the logo (sketchy) of the Department of Computer Science of the University of Kaiserslautern (black when in grayscale mode)
\iftoggle{bwmode}{
	\definecolor{ce5e8f5}{RGB}{0,0,0}
	\definecolor{cdfdbe2}{RGB}{0,0,0}
	\definecolor{cafcde9}{RGB}{0,0,0}
	\definecolor{c9c9afc}{RGB}{0,0,0}
	\definecolor{cb9babc}{RGB}{0,0,0}
	\definecolor{cddae9b}{RGB}{0,0,0}
	\definecolor{cf8776f}{RGB}{0,0,0}
	\definecolor{cac9b8b}{RGB}{0,0,0}
	\definecolor{c878cb9}{RGB}{0,0,0}
	\definecolor{c848387}{RGB}{0,0,0}
	\definecolor{c6c6898}{RGB}{0,0,0}
	\definecolor{cf52d21}{RGB}{0,0,0}
	\definecolor{c0503fc}{RGB}{0,0,0}
	\definecolor{cfa0305}{RGB}{0,0,0}
}{
	\definecolor{ce5e8f5}{RGB}{229,232,245}
	\definecolor{cdfdbe2}{RGB}{223,219,226}
	\definecolor{cafcde9}{RGB}{175,205,233}
	\definecolor{c9c9afc}{RGB}{156,154,252}
	\definecolor{cb9babc}{RGB}{185,186,188}
	\definecolor{cddae9b}{RGB}{221,174,155}
	\definecolor{cf8776f}{RGB}{248,119,111}
	\definecolor{cac9b8b}{RGB}{172,155,139}
	\definecolor{c878cb9}{RGB}{135,140,185}
	\definecolor{c848387}{RGB}{132,131,135}
	\definecolor{c6c6898}{RGB}{108,104,152}
	\definecolor{cf52d21}{RGB}{245,45,33}
	\definecolor{c0503fc}{RGB}{5,3,252}
	\definecolor{cfa0305}{RGB}{250,3,5}
}

% The logo (sketchy) of the Department of Computer Science of the University of Kaiserslautern:
% Taken from: http://dekanat.informatik.uni-kl.de/logo_dekanat_400x145.png on the 2016-12-14
% Converted to SVG using: vectorizer (https://www.vectorizer.io/)
% Manipulated using: Inkscape (https://inkscape.org/)
% Converted to TikZ using: svg2tikz (https://github.com/kjellmf/svg2tikz) as an Inkscape extension
\newcommand{\CSLogoSketchy}{
	\begin{tikzpicture}[
		y = 0.1pt,
		x = 0.1pt,
		yscale = -1,
		xscale = 1,
	]
		\begin{scope}[fill = ce5e8f5]
			\path[fill] (1043,1330) .. controls (1043,1305) and   (1045,1295) .. (1047,1308) .. controls   (1049,1320) and (1049,1340) .. (1047,1353) ..   controls (1045,1365) and (1043,1355) ..   (1043,1330) -- cycle;
			\path[fill] (1043,1050) .. controls (1043,1020) and   (1045,1007) .. (1047,1023) .. controls   (1049,1038) and (1049,1062) .. (1047,1078) ..   controls (1045,1093) and (1043,1080) ..   (1043,1050) -- cycle;
			\path[fill] (780,890) .. controls (767,882) and   (768,880) .. (783,880) .. controls (792,880) and   (800,885) .. (800,890) .. controls (800,902) and   (799,902) .. (780,890) -- cycle;
			\path[fill] (774,640) .. controls (774,516) and   (776,466) .. (777,528) .. controls (779,589) and   (779,691) .. (777,753) .. controls (776,814) and   (774,764) .. (774,640) -- cycle;
			\path[fill] (838,853) .. controls (844,851) and   (856,851) .. (863,853) .. controls (869,856) and   (864,858) .. (850,858) .. controls (836,858) and   (831,856) .. (838,853) -- cycle;
			\path[fill] (1043,715) .. controls (1043,682) and   (1045,670) .. (1047,688) .. controls (1049,706)   and (1049,733) .. (1047,748) .. controls   (1045,763) and (1043,748) .. (1043,715) --   cycle;
			\path[fill] (1043,420) .. controls (1043,384) and   (1045,370) .. (1047,388) .. controls (1049,405)   and (1049,435) .. (1047,453) .. controls   (1045,470) and (1043,456) .. (1043,420) --   cycle;
			\path[fill] (133,265) .. controls (133,221) and   (135,204) .. (137,228) .. controls (139,251) and   (139,287) .. (137,308) .. controls (135,328) and   (133,309) .. (133,265) -- cycle;
		\end{scope}
		\begin{scope}[fill = cdfdbe2]
			\path[fill] (933,1283) .. controls (942,1281) and (958,1281) .. (968,1283) .. controls (977,1286) and (969,1288) .. (950,1288) .. controls (931,1288) and (923,1286) .. (933,1283) -- cycle;
			\path[fill] (1090,1283) -- (1129,1279) -- (1133,1202) -- (1136,1125) -- (1136,1205) -- (1135,1285) -- (1092,1286) -- (1050,1287) -- (1090,1283) -- cycle;
			\path[fill] (909,923) .. controls (896,907) and (897,906) .. (913,919) .. controls (922,926) and (930,934) .. (930,936) .. controls (930,944) and (922,939) .. (909,923) -- cycle;
			\path[fill] (823,903) .. controls (838,901) and (860,901) .. (873,903) .. controls (885,905) and (873,907) .. (845,907) .. controls (818,907) and (807,905) .. (823,903) -- cycle;
			\path[fill] (880,846) .. controls (880,844) and (888,836) .. (898,829) .. controls (913,816) and (914,817) .. (901,833) .. controls (888,849) and (880,854) .. (880,846) -- cycle;
			\path[fill] (1133,580) .. controls (1133,533) and (1135,514) .. (1137,538) .. controls (1139,561) and (1139,599) .. (1137,623) .. controls (1135,646) and (1133,627) .. (1133,580) -- cycle;
			\path[fill=cdfdbe2] (828,393) .. controls (856,391) and (904,391) .. (933,393) .. controls (961,395) and (938,396) .. (880,396) .. controls (822,396) and (799,395) .. (828,393) -- cycle;
			\end{scope}
		\begin{scope}[fill = cafcde9]
			\path[fill] (328,383) .. controls (356,381) and (404,381) .. (433,383) .. controls (461,385) and (438,386) .. (380,386) .. controls (322,386) and (299,385) .. (328,383) -- cycle;
			\path[fill] (678,13) .. controls (685,10) and (694,11) .. (697,14) .. controls (701,17) and (695,20) .. (684,19) .. controls (673,19) and (670,16) .. (678,13) -- cycle;
		\end{scope}
		\begin{scope}[fill = c9c9afc]
			\path[fill] (993,1343) .. controls (985,1341) and (980,1321) .. (980,1299) .. controls (980,1267) and (983,1260) .. (1000,1260) .. controls (1017,1260) and (1020,1267) .. (1020,1305) .. controls (1020,1349) and (1017,1353) .. (993,1343) -- cycle;
			\path[fill] (56,1188) .. controls (79,1104) and (103,1019) .. (109,1000) .. controls (115,981) and (148,866) .. (181,745) .. controls (214,624) and (250,495) .. (261,458) -- (281,390) -- (381,390) .. controls (458,390) and (481,393) .. (477,403) .. controls (475,409) and (436,543) .. (390,700) .. controls (345,857) and (285,1064) .. (256,1160) -- (204,1335) -- (108,1338) -- (13,1341) -- (56,1188) -- cycle;
			\path[fill] (980,1046) .. controls (980,997) and (982,992) .. (1000,997) .. controls (1017,1001) and (1020,1011) .. (1020,1051) .. controls (1020,1093) and (1017,1100) .. (1000,1100) .. controls (982,1100) and (980,1093) .. (980,1046) -- cycle;
			\path[fill] (780,640) -- (780,400) -- (880,400) -- (980,400) -- (980,365) -- (980,330) -- (560,330) -- (140,330) -- (140,257) -- (140,183) -- (363,115) .. controls (485,78) and (608,41) .. (636,32) -- (688,17) -- (891,79) .. controls (1003,113) and (1127,151) .. (1165,163) -- (1235,185) -- (1238,258) -- (1241,330) -- (1130,330) -- (1020,330) -- (1020,405) .. controls (1020,473) and (1018,480) .. (1000,480) .. controls (984,480) and (980,473) .. (980,445) -- (980,410) -- (890,410) -- (800,410) -- (800,635) -- (800,860) -- (830,860) .. controls (847,860) and (860,865) .. (860,870) .. controls (860,876) and (842,880) .. (820,880) -- (780,880) -- (780,640) -- cycle;
			\path[fill] (980,688) .. controls (980,647) and (983,640) .. (1000,640) .. controls (1017,640) and (1020,647) .. (1020,688) .. controls (1020,729) and (1017,736) .. (1000,736) .. controls (983,736) and (980,729) .. (980,688) -- cycle;
		\end{scope}
		\begin{scope}[fill = cb9babc]
			\path[fill] (868,433) .. controls (891,431) and (927,431) .. (948,433) .. controls (968,435) and (949,437) .. (905,437) .. controls (861,437) and (844,435) .. (868,433) -- cycle;
			\path[fill] (363,353) .. controls (477,351) and (663,351) .. (778,353) .. controls (892,354) and (798,355) .. (570,355) .. controls (342,355) and (248,354) .. (363,353) -- cycle;
			\path[fill] (1093,353) .. controls (1124,351) and (1176,351) .. (1208,353) .. controls (1239,355) and (1213,356) .. (1150,356) .. controls (1087,356) and (1061,355) .. (1093,353) -- cycle;
		\end{scope}
		\begin{scope}[fill = cddae9b]
			\path[fill] (1080,815) .. controls (1056,790) and (1038,770) .. (1041,770) .. controls (1044,770) and (1066,790) .. (1090,815) .. controls (1114,840) and (1132,860) .. (1129,860) .. controls (1126,860) and (1104,840) .. (1080,815) -- cycle;
		\end{scope}
		\begin{scope}[fill = cf8776f]
			\path[fill] (959,963) -- (935,935) -- (963,959) .. controls (988,982) and (995,990) .. (987,990) .. controls (985,990) and (973,978) .. (959,963) -- cycle;
		\end{scope}
		\begin{scope}[fill = cac9b8b]
			\path[fill] (1090,945) .. controls (1120,915) and (1147,890) .. (1149,890) .. controls (1152,890) and (1130,915) .. (1100,945) .. controls (1070,975) and (1043,1000) .. (1041,1000) .. controls (1038,1000) and (1060,975) .. (1090,945) -- cycle;
		\end{scope}
		\begin{scope}[fill = c878cb9]
			\path[fill] (77,1343) .. controls (101,1341) and (139,1341) .. (162,1343) .. controls (186,1345) and (167,1347) .. (120,1347) .. controls (73,1347) and (54,1345) .. (77,1343) -- cycle;
			\path[fill] (805,635) -- (805,415) -- (890,414) -- (975,414) -- (893,417) -- (810,421) -- (807,638) -- (804,855) -- (805,635) -- cycle;
			\path[fill] (363,333) .. controls (477,331) and (663,331) .. (778,333) .. controls (892,334) and (798,335) .. (570,335) .. controls (342,335) and (248,334) .. (363,333) -- cycle;
			\path[fill] (1073,333) .. controls (1104,331) and (1156,331) .. (1188,333) .. controls (1219,335) and (1193,336) .. (1130,336) .. controls (1067,336) and (1041,335) .. (1073,333) -- cycle;
		\end{scope}
		\begin{scope}[fill = c848387]
			\path[fill] (1014,1354) .. controls (1017,1345) and (1020,1323) .. (1020,1304) .. controls (1020,1270) and (1020,1270) .. (1065,1270) -- (1110,1270) -- (1110,1195) .. controls (1110,1152) and (1114,1120) .. (1120,1120) .. controls (1126,1120) and (1130,1153) .. (1130,1200) -- (1130,1280) -- (1085,1280) -- (1040,1280) -- (1040,1325) .. controls (1040,1360) and (1036,1370) .. (1024,1370) .. controls (1013,1370) and (1010,1365) .. (1014,1354) -- cycle;
			\path[fill] (113,1353) .. controls (193,1350) and (202,1347) .. (210,1327) .. controls (219,1304) and (247,1211) .. (393,705) .. controls (439,546) and (482,414) .. (489,412) .. controls (495,410) and (500,412) .. (500,417) .. controls (500,425) and (422,694) .. (274,1198) -- (227,1360) -- (126,1358) -- (25,1356) -- (113,1353) -- cycle;
			\path[fill] (933,1273) .. controls (942,1271) and (958,1271) .. (968,1273) .. controls (977,1276) and (969,1278) .. (950,1278) .. controls (931,1278) and (923,1276) .. (933,1273) -- cycle;
			\path[fill] (1020,1042) .. controls (1020,985) and (1021,984) .. (1079,926) .. controls (1111,894) and (1140,872) .. (1143,877) .. controls (1146,882) and (1125,911) .. (1095,940) .. controls (1042,992) and (1040,996) .. (1040,1047) .. controls (1040,1076) and (1036,1100) .. (1030,1100) .. controls (1024,1100) and (1020,1074) .. (1020,1042) -- cycle;
			\path[fill] (800,890) .. controls (800,885) and (815,880) .. (834,880) .. controls (853,880) and (872,885) .. (875,890) .. controls (879,896) and (865,900) .. (841,900) .. controls (818,900) and (800,896) .. (800,890) -- cycle;
			\path[fill] (812,643) -- (810,420) -- (898,422) -- (985,424) -- (903,427) -- (820,431) -- (817,648) -- (815,865) -- (812,643) -- cycle;
			\path[fill] (1028,754) .. controls (1023,750) and (1020,723) .. (1020,693) -- (1020,640) -- (1065,640) -- (1110,640) -- (1110,575) .. controls (1110,538) and (1114,510) .. (1120,510) .. controls (1126,510) and (1130,542) .. (1130,585) -- (1130,660) -- (1086,660) -- (1041,660) -- (1038,711) .. controls (1036,739) and (1032,758) .. (1028,754) -- cycle;
			\path[fill] (920,650) .. controls (920,645) and (934,640) .. (950,640) .. controls (967,640) and (980,645) .. (980,650) .. controls (980,656) and (967,660) .. (950,660) .. controls (934,660) and (920,656) .. (920,650) -- cycle;
			\path[fill] (1020,410) -- (1020,340) -- (1130,340) -- (1240,340) -- (1240,270) .. controls (1240,230) and (1244,200) .. (1250,200) .. controls (1256,200) and (1260,232) .. (1260,275) -- (1260,350) -- (1150,350) -- (1040,350) -- (1040,415) .. controls (1040,452) and (1036,480) .. (1030,480) .. controls (1024,480) and (1020,450) .. (1020,410) -- cycle;
			\path[fill] (363,343) .. controls (477,341) and (663,341) .. (778,343) .. controls (892,344) and (798,345) .. (570,345) .. controls (342,345) and (248,344) .. (363,343) -- cycle;
		\end{scope}
		\begin{scope}[fill = c6c6898]
			\path[fill] (933,1263) .. controls (942,1261) and (958,1261) .. (968,1263) .. controls (977,1266) and (969,1268) .. (950,1268) .. controls (931,1268) and (923,1266) .. (933,1263) -- cycle;
			\path[fill] (1043,1263) .. controls (1058,1261) and (1080,1261) .. (1093,1263) .. controls (1105,1265) and (1093,1267) .. (1065,1267) .. controls (1038,1267) and (1027,1265) .. (1043,1263) -- cycle;
		\end{scope}
		\begin{scope}[fill = cf52d21]
			\path[fill] (1065,930) .. controls (1098,897) and (1127,870) .. (1129,870) .. controls (1132,870) and (1108,897) .. (1075,930) .. controls (1042,963) and (1013,990) .. (1011,990) .. controls (1008,990) and (1032,963) .. (1065,930) -- cycle;
			\path[fill] (914,918) -- (895,895) -- (918,914) .. controls (939,932) and (945,940) .. (937,940) .. controls (935,940) and (925,930) .. (914,918) -- cycle;
		\end{scope}
		\begin{scope}[fill = c0503fc]
			\path[fill] (890,1180) -- (890,1100) -- (994,1100) .. controls (1114,1100) and (1110,1097) .. (1110,1196) -- (1110,1260) -- (1000,1260) -- (890,1260) -- (890,1180) -- cycle;
			\path[fill] (890,560) -- (890,480) -- (994,480) .. controls (1114,480) and (1110,477) .. (1110,576) -- (1110,640) -- (1000,640) -- (890,640) -- (890,560) -- cycle;
		\end{scope}
		\begin{scope}[fill = cfa0305]
			\path[fill] (932,933) -- (869,868) -- (934,802) -- (1000,735) -- (1064,800) -- (1129,865) -- (1064,933) .. controls (1027,970) and (997,999) .. (996,998) .. controls (996,998) and (967,968) .. (932,933) -- cycle;
		\end{scope}
	
	\end{tikzpicture}
}

% Colors needed for the logo (sketchy) of the Department of Computer Science of the University of Kaiserslautern (black when in grayscale mode)
\iftoggle{bwmode}{
	\definecolor{c808080}{RGB}{191,191,191}
	\definecolor{cff0000}{RGB}{127,127,127}
	\definecolor{c9999ff}{RGB}{0,0,0}
	\definecolor{c0000ff}{RGB}{63,63,63}
}{
	\definecolor{c808080}{RGB}{128,128,128}
	\definecolor{cff0000}{RGB}{255,0,0}
	\definecolor{c9999ff}{RGB}{153,153,255}
	\definecolor{c0000ff}{RGB}{0,0,255}
}

% The logo of the Department of Computer Science of the University of Kaiserslautern:
% Taken from: http://sci.informatik.uni-kl.de/rechnerzugang/terminals/lageplan_sci/Lageplan_SCI.pdf on the 2017-03-16
% Manipulated using: Inkscape (https://inkscape.org/)
% Converted to TikZ using: svg2tikz (https://github.com/kjellmf/svg2tikz) as an Inkscape extension
\newcommand{\CSLogo}{
	\begin{tikzpicture}[
		y = 1.65pt,
		x = 1.65pt,
		yscale = -1,
		xscale = 1
	]
		\begin{scope}[cm={{0.0, 1.25, 1.25, 0.0, (-153.75, -108.75)}}]
			\path[cm = {{0.0, 0.82808, 1.0, 0.0, (125.0216, 604.3906)}}, fill = c808080, nonzero rule] (0.0000, 0.0000) node[above right] (text2307) {};
			\path[cm = {{0.0, 0.82379, 1.0, 0.0, (125.0216, 609.1211)}}, fill = c808080, nonzero rule] (0.0000, 0.0000) node[above right] (text2311) {};
			\path[cm = {{0.0, 0.82808, 1.0, 0.0, (145.8526, 604.3906)}}, fill = c808080, nonzero rule] (0.0000, 0.0000) node[above right] (text2315) {};
			\path[cm = {{0.0, 0.82379, 1.0, 0.0, (145.8526, 609.1211)}}, fill = c808080, nonzero rule] (0.0000, 0.0000) node[above right] (text2319) {};
			\path[cm = {{0.0, 0.75, 1.0, 0.0, (168.3434, 604.3906)}}, fill = c808080, nonzero rule] (0.0000, 0.0000) node[above right] (text2323) {};
			\path[cm = {{0.0, 1.0, 0.97692, 0.0, (125.1046, 587.6261)}}, fill = c808080, nonzero rule] (0.0000, 0.0000) node[above right] (text2327) {};
			\path[cm = {{0.0, 1.0, 0.95143, 0.0, (147.0145, 587.3771)}}, fill = c808080, nonzero rule] (0.0000, 0.0000) node[above right] (text2331) {};
			\path[cm = {{0.0, 1.0, 0.62474, 0.0, (169.6713, 591.1118)}}, fill = c808080, nonzero rule] (0.0000, 0.0000) node[above right] (text2355) {};
			\path[cm = {{0.0, 1.0, 0.97692, 0.0, (123.8597, 585.9662)}}, fill = cff0000, nonzero rule] (0.0000, 0.0000) node[above right] (text2379) {};
			\path[cm = {{0.0, 1.0, 0.95143, 0.0, (145.8526, 585.7173)}}, fill = cff0000, nonzero rule] (0.0000, 0.0000) node[above right] (text2383) {};
			\path[cm = {{0.0, 1.0, 0.62474, 0.0, (168.5094, 589.452)}}, fill = cff0000, nonzero rule] (0.0000, 0.0000) node[above right] (text2407) {};
			\path[fill = c808080, even odd rule] (123.5280, 563.8070) -- (123.5280, 558.9940) -- (122.4490, 558.9940) -- (122.4490, 568.6210) -- (123.5280, 568.6210) -- (123.5280, 563.8070);
			\path[fill = c808080, even odd rule] (144.9400, 563.8070) -- (144.9400, 558.9940) -- (143.7780, 558.9940) -- (143.7780, 568.6210) -- (144.9400, 568.6210) -- (144.9400, 563.8070);
			\path[fill = c808080, even odd rule] (133.7360, 559.0770) -- (122.4490, 559.0770) -- (122.4490, 560.2390) -- (145.0230, 560.2390) -- (145.0230, 559.0770) -- (133.7360, 559.0770);
			\path[fill = c808080, even odd rule] (143.1970, 568.6210) -- (119.7100, 568.6210) -- (119.7100, 570.3640) -- (166.6010, 570.3640) -- (166.6010, 568.6210) -- (143.1970, 568.6210);
			\path[fill = c808080, even odd rule] (112.4070, 529.1170) -- (104.6890, 554.7610) -- (112.4070, 580.5720) -- (119.7100, 580.5720) -- (119.7100, 529.2830) -- (112.4070, 529.2830) -- (112.5730, 529.1170) -- (112.4070, 529.1170);
			\path[fill = c808080, even odd rule] (122.1170, 535.7560) -- (122.1170, 545.3830) -- (166.4350, 532.4360) -- (166.4350, 523.2240) -- (122.1170, 535.7560);
			\path[fill = c808080, even odd rule] (147.2630, 566.2970) -- (144.2760, 563.2260) -- (138.1340, 569.4510) -- (144.1930, 575.5920) -- (150.3340, 569.3680) -- (147.2630, 566.2970);
			\path[fill = c808080, even odd rule] (133.8190, 569.4510) -- (133.8190, 564.3880) -- (126.4320, 564.3880) -- (126.4320, 574.4300) -- (133.8190, 574.4300) -- (133.8190, 569.4510);
			\path[fill = c808080, even odd rule] (162.6170, 569.4510) -- (162.6170, 564.3880) -- (155.1480, 564.3880) -- (155.1480, 574.4300) -- (162.6170, 574.4300) -- (162.6170, 569.4510);
			\path[fill = c9999ff, even odd rule] (122.6150, 562.8110) -- (122.6150, 557.9150) -- (121.5360, 557.9150) -- (121.5360, 567.6250) -- (122.6150, 567.6250) -- (122.6150, 562.8110);
			\path[fill = c9999ff, even odd rule] (144.0270, 562.8110) -- (144.0270, 557.9150) -- (142.8650, 557.9150) -- (142.8650, 567.6250) -- (144.0270, 567.6250) -- (144.0270, 562.8110);
			\path[fill = c9999ff, even odd rule] (132.8230, 557.9980) -- (121.5360, 557.9980) -- (121.5360, 559.1600) -- (144.1100, 559.1600) -- (144.1100, 557.9980) -- (132.8230, 557.9980);
			\path[fill = c9999ff, even odd rule] (142.2010, 567.6250) -- (118.7140, 567.6250) -- (118.7140, 569.2850) -- (165.6880, 569.2850) -- (165.6880, 567.6250) -- (142.2010, 567.6250);
			\path[fill = c9999ff, even odd rule] (111.4940, 528.0380) -- (103.6930, 553.6820) -- (111.4940, 579.4930) -- (118.7140, 579.4930) -- (118.7140, 528.2040) -- (111.4940, 528.2040) -- (111.6600, 528.0380) -- (111.4940, 528.0380);
			\path[fill = c9999ff, even odd rule] (121.1210, 534.6770) -- (121.1210, 544.3040) -- (165.5220, 531.3570) -- (165.5220, 522.1450) -- (121.1210, 534.6770);
			\path[fill = cff0000, even odd rule] (146.3510, 565.2180) -- (143.2800, 562.1480) -- (137.1380, 568.3720) -- (143.2800, 574.5130) -- (149.4210, 568.2890) -- (146.3510, 565.2180);
			\path[fill = c0000ff, even odd rule] (132.9060, 568.3720) -- (132.9060, 563.3090) -- (125.4370, 563.3090) -- (125.4370, 573.4340) -- (132.9060, 573.4340) -- (132.9060, 568.3720);
			\path[fill = c0000ff, even odd rule] (161.7040, 568.3720) -- (161.7040, 563.3090) -- (154.2350, 563.3090) -- (154.2350, 573.4340) -- (161.7040, 573.4340) -- (161.7040, 568.3720);
			\path[cm = {{0.0, 0.82808, 1.0, 0.0, (124.0257, 603.3947)}}, fill = c9999ff, nonzero rule] (0.0000, 0.0000) node[above right] (text2467) {};
			\path[cm = {{0.0, 0.82379, 1.0, 0.0, (124.0257, 608.1252)}}, fill = c9999ff, nonzero rule] (0.0000, 0.0000) node[above right] (text2471) {};
			\path[cm = {{0.0, 0.82808, 1.0, 0.0, (144.8567, 603.3947)}}, fill = c9999ff, nonzero rule] (0.0000, 0.0000) node[above right] (text2475) {};
			\path[cm = {{0.0, 0.82379, 1.0, 0.0, (144.8567, 608.1252)}}, fill = c9999ff, nonzero rule] (0.0000, 0.0000) node[above right] (text2479) {};
			\path[cm = {{0.0, 0.75, 1.0, 0.0, (167.4305, 603.3947)}}, fill = c0000ff, nonzero rule] (0.0000, 0.0000) node[above right] (text2483) {};
		\end{scope}
		
	\end{tikzpicture}
}


%%%%%%%%%%%%%%%%%%%%%%%%%%%%%%%%%%%%%%%%%%%%%%%
%%%%%%%%%%%%%%%%%%%%%%%%%%%%%%%%%%%%%%%%%%%%%%%
%%%%%%%%%%%%%%%%%%%%%%%%%%%%%%%%%%%%%%%%%%%%%%%
% Defines a shape 'square' for tikz that behaves like a square :-):
%%%%%%%%%%%%%%%%%%%%%%%%%%%%%%%%%%%%%%%%%%%%%%% begindefinition
\makeatletter
% the contents of \squarecorner were mostly stolen from pgfmoduleshapes.code.tex
\def\squarecorner#1{
    % Calculate x
    %
    % First, is width < minimum width?
    \pgf@x=\the\wd\pgfnodeparttextbox%
    \pgfmathsetlength\pgf@xc{\pgfkeysvalueof{/pgf/inner xsep}}%
    \advance\pgf@x by 2\pgf@xc%
    \pgfmathsetlength\pgf@xb{\pgfkeysvalueof{/pgf/minimum width}}%
    \ifdim\pgf@x<\pgf@xb%
        % yes, too small. Enlarge...
        \pgf@x=\pgf@xb%
    \fi%
    % Calculate y
    %
    % First, is height+depth < minimum height?
    \pgf@y=\ht\pgfnodeparttextbox%
    \advance\pgf@y by\dp\pgfnodeparttextbox%
    \pgfmathsetlength\pgf@yc{\pgfkeysvalueof{/pgf/inner ysep}}%
    \advance\pgf@y by 2\pgf@yc%
    \pgfmathsetlength\pgf@yb{\pgfkeysvalueof{/pgf/minimum height}}%
    \ifdim\pgf@y<\pgf@yb%
        % yes, too small. Enlarge...
        \pgf@y=\pgf@yb%
    \fi%
    %
    % this \ifdim is the actual part that makes the node dimensions square.
    \ifdim\pgf@x<\pgf@y%
        \pgf@x=\pgf@y%
    \else
        \pgf@y=\pgf@x%
    \fi
    %
    % Now, calculate right border: .5\wd\pgfnodeparttextbox + .5 \pgf@x + #1outer sep
    \pgf@x=#1.5\pgf@x%
    \advance\pgf@x by.5\wd\pgfnodeparttextbox%
    \pgfmathsetlength\pgf@xa{\pgfkeysvalueof{/pgf/outer xsep}}%
    \advance\pgf@x by#1\pgf@xa%
    % Now, calculate upper border: .5\ht-.5\dp + .5 \pgf@y + #1outer sep
    \pgf@y=#1.5\pgf@y%
    \advance\pgf@y by-.5\dp\pgfnodeparttextbox%
    \advance\pgf@y by.5\ht\pgfnodeparttextbox%
    \pgfmathsetlength\pgf@ya{\pgfkeysvalueof{/pgf/outer ysep}}%
    \advance\pgf@y by#1\pgf@ya%
}
\makeatother

\pgfdeclareshape{simplesquare}{
    \savedanchor\northeast{\squarecorner{}}
    \savedanchor\southwest{\squarecorner{-}}

    \foreach \x in {east,west} \foreach \y in {north,mid,base,south} {
        \inheritanchor[from=rectangle]{\y\space\x}
    }
    \foreach \x in {east,west,north,mid,base,south,center,text} {
        \inheritanchor[from=rectangle]{\x}
    }
    \inheritanchorborder[from=rectangle]
    \inheritbackgroundpath[from=rectangle]
}
%%%%%%%%%%%%%%%%%%%%%%%%%%%%%%%%%%%%%%%%%%%%%%% enddefinition

%%%%%%%%%%%%%%%%%%%%%%%%%%%%%%%%%%%%%%%%%%%%%%%
%%%%%%%%%%%%%%%%%%%%%%%%%%%%%%%%%%%%%%%%%%%%%%%
%%%%%%%%%%%%%%%%%%%%%%%%%%%%%%%%%%%%%%%%%%%%%%%
% Adds 'west north west', 'east north east', 'east south east', 'north north west', 
% 'south south west', 'south south east' to the tikz shape 'rectangle':
%%%%%%%%%%%%%%%%%%%%%%%%%%%%%%%%%%%%%%%%%%%%%%% begindefinition
\makeatletter
\pgfdeclareshape{square}{
  \inheritsavedanchors[from=simplesquare]
  \inheritanchorborder[from=simplesquare]
  \foreach \a in {%
      center,mid,base,north,south,west,east,%
      north west,mid west,base west,south west,%
      north east,mid east,base east,south east%
    }{\inheritanchor[from=simplesquare]{\a}}
  \inheritbackgroundpath[from=simplesquare]
  \anchor{north 1/3}{
    \southwest\pgf@xa=\pgf@x
    \northeast\pgfmathsetlength\pgf@x{\pgf@xa-(\pgf@xa-\pgf@x)/3}
  }
  \anchor{north 2/3}{
    \southwest\pgf@xa=\pgf@x
    \northeast\pgfmathsetlength\pgf@x{\pgf@x-(\pgf@x-\pgf@xa)/3}
  }
  \anchor{south 1/3}{
    \northeast\pgf@xa=\pgf@x
    \southwest\pgfmathsetlength\pgf@x{\pgf@x-(\pgf@x-\pgf@xa)/3}
  }
  \anchor{south 2/3}{
    \northeast\pgf@xa=\pgf@x
    \southwest\pgfmathsetlength\pgf@x{\pgf@xa-(\pgf@xa-\pgf@x)/3}
  }
  \anchor{east 1/3}{
    \southwest\pgf@ya=\pgf@y
    \northeast\pgfmathsetlength\pgf@y{\pgf@ya-(\pgf@ya-\pgf@y)/3}
  }
  \anchor{east 2/3}{
    \southwest\pgf@ya=\pgf@y
    \northeast\pgfmathsetlength\pgf@y{\pgf@y-(\pgf@y-\pgf@ya)/3}
  }
  \anchor{west 1/3}{
    \northeast\pgf@ya=\pgf@y
    \southwest\pgfmathsetlength\pgf@y{\pgf@y-(\pgf@y-\pgf@ya)/3}
  }
  \anchor{west 2/3}{
    \northeast\pgf@ya=\pgf@y
    \southwest\pgfmathsetlength\pgf@y{\pgf@ya-(\pgf@ya-\pgf@y)/3}
  }
}
\makeatother
%%%%%%%%%%%%%%%%%%%%%%%%%%%%%%%%%%%%%%%%%%%%%%% enddefinition

%%%%%%%%%%%%%%%%%%%%%%%%%%%%%%%%%%%%%%%%%%%%%%%
%%%%%%%%%%%%%%%%%%%%%%%%%%%%%%%%%%%%%%%%%%%%%%%
%%%%%%%%%%%%%%%%%%%%%%%%%%%%%%%%%%%%%%%%%%%%%%%
% Redefines the bibliography entry for inproceedings:
%%%%%%%%%%%%%%%%%%%%%%%%%%%%%%%%%%%%%%%%%%%%%%% begindefinition
\DeclareBibliographyDriver{inproceedings}{%
  \usebibmacro{bibindex}%
  \usebibmacro{begentry}%
  \usebibmacro{author}%
  \setunit{\printdelim{nametitledelim}}\newblock
  \usebibmacro{title}%
  \newunit\newblock
  \usebibmacro{date}%
  \usebibmacro{finentry}}
%%%%%%%%%%%%%%%%%%%%%%%%%%%%%%%%%%%%%%%%%%%%%%% enddefinition

%%%%%%%%%%%%%%%%%%%%%%%%%%%%%%%%%%%%%%%%%%%%%%%
%%%%%%%%%%%%%%%%%%%%%%%%%%%%%%%%%%%%%%%%%%%%%%%
%%%%%%%%%%%%%%%%%%%%%%%%%%%%%%%%%%%%%%%%%%%%%%%
% Redefines the bibliography entry for article:
%%%%%%%%%%%%%%%%%%%%%%%%%%%%%%%%%%%%%%%%%%%%%%% begindefinition
\DeclareBibliographyDriver{article}{%
  \usebibmacro{bibindex}%
  \usebibmacro{begentry}%
  \usebibmacro{author}%
  \setunit{\printdelim{nametitledelim}}\newblock
  \usebibmacro{title}%
  \newunit\newblock
  \usebibmacro{date}%
  \usebibmacro{finentry}}
%%%%%%%%%%%%%%%%%%%%%%%%%%%%%%%%%%%%%%%%%%%%%%% enddefinition

%%%%%%%%%%%%%%%%%%%%%%%%%%%%%%%%%%%%%%%%%%%%%%%
%%%%%%%%%%%%%%%%%%%%%%%%%%%%%%%%%%%%%%%%%%%%%%%
%%%%%%%%%%%%%%%%%%%%%%%%%%%%%%%%%%%%%%%%%%%%%%%
% Redefines the bibliography entry for book:
%%%%%%%%%%%%%%%%%%%%%%%%%%%%%%%%%%%%%%%%%%%%%%% begindefinition
\DeclareBibliographyDriver{book}{%
  \usebibmacro{bibindex}%
  \usebibmacro{begentry}%
  \usebibmacro{author/editor+others/translator+others}%
  \setunit{\printdelim{nametitledelim}}\newblock
  \usebibmacro{maintitle+title}%
  \newunit\newblock
  \usebibmacro{date}%
  \newunit\newblock
  \newunit\newblock
  \iftoggle{bbx:isbn}
    {\printfield{isbn}}
    {}%
  \usebibmacro{finentry}}
%%%%%%%%%%%%%%%%%%%%%%%%%%%%%%%%%%%%%%%%%%%%%%% enddefinition

%%%%%%%%%%%%%%%%%%%%%%%%%%%%%%%%%%%%%%%%%%%%%%%
%%%%%%%%%%%%%%%%%%%%%%%%%%%%%%%%%%%%%%%%%%%%%%%
%%%%%%%%%%%%%%%%%%%%%%%%%%%%%%%%%%%%%%%%%%%%%%%
% Redefines the bibliography entry for misc:
%%%%%%%%%%%%%%%%%%%%%%%%%%%%%%%%%%%%%%%%%%%%%%% begindefinition
\DeclareBibliographyDriver{misc}{%
  \usebibmacro{bibindex}%
  \usebibmacro{begentry}%
  \usebibmacro{author/editor+others/translator+others}%
  \setunit{\printdelim{nametitledelim}}\newblock
  \usebibmacro{title}%
  \newunit\newblock
  \printfield{type}%
  \newunit\newblock
  \usebibmacro{date}%
  \newunit\newblock
  \usebibmacro{url}%
  \usebibmacro{finentry}}
%%%%%%%%%%%%%%%%%%%%%%%%%%%%%%%%%%%%%%%%%%%%%%% enddefinition

%%%%%%%%%%%%%%%%%%%%%%%%%%%%%%%%%%%%%%%%%%%%%%%
%%%%%%%%%%%%%%%%%%%%%%%%%%%%%%%%%%%%%%%%%%%%%%%
%%%%%%%%%%%%%%%%%%%%%%%%%%%%%%%%%%%%%%%%%%%%%%%
% Redefines the bibliography entry for online:
%%%%%%%%%%%%%%%%%%%%%%%%%%%%%%%%%%%%%%%%%%%%%%% begindefinition
\DeclareBibliographyDriver{online}{%
  \usebibmacro{bibindex}%
  \usebibmacro{begentry}%
  \usebibmacro{author/editor+others/translator+others}%
  \setunit{\printdelim{nametitledelim}}\newblock
  \usebibmacro{title}%
  \newunit\newblock
  \usebibmacro{date}%
  \newunit\newblock
  \usebibmacro{url+urldate}%
  \usebibmacro{finentry}}
%%%%%%%%%%%%%%%%%%%%%%%%%%%%%%%%%%%%%%%%%%%%%%% enddefinition

%%%%%%%%%%%%%%%%%%%%%%%%%%%%%%%%%%%%%%%%%%%%%%%
%%%%%%%%%%%%%%%%%%%%%%%%%%%%%%%%%%%%%%%%%%%%%%%
%%%%%%%%%%%%%%%%%%%%%%%%%%%%%%%%%%%%%%%%%%%%%%%
% Redefines the bibliography entry for proceedings:
%%%%%%%%%%%%%%%%%%%%%%%%%%%%%%%%%%%%%%%%%%%%%%% begindefinition
\DeclareBibliographyDriver{proceedings}{%
  \usebibmacro{bibindex}%
  \usebibmacro{begentry}%
  \usebibmacro{editor+others}%
  \setunit{\printdelim{nametitledelim}}\newblock
  \usebibmacro{maintitle+title}%
  \newunit\newblock
  \usebibmacro{date}%
  \usebibmacro{finentry}}
%%%%%%%%%%%%%%%%%%%%%%%%%%%%%%%%%%%%%%%%%%%%%%% enddefinition

%%%%%%%%%%%%%%%%%%%%%%%%%%%%%%%%%%%%%%%%%%%%%%%
%%%%%%%%%%%%%%%%%%%%%%%%%%%%%%%%%%%%%%%%%%%%%%%
%%%%%%%%%%%%%%%%%%%%%%%%%%%%%%%%%%%%%%%%%%%%%%%
% Redefines the bibliography entry for report:
%%%%%%%%%%%%%%%%%%%%%%%%%%%%%%%%%%%%%%%%%%%%%%% begindefinition
\DeclareBibliographyDriver{report}{%
  \usebibmacro{bibindex}%
  \usebibmacro{begentry}%
  \usebibmacro{author}%
  \usebibmacro{title}%
  \newunit\newblock
  \printfield{type}%
  \newunit\newblock
  \usebibmacro{date}%
  \usebibmacro{finentry}}
%%%%%%%%%%%%%%%%%%%%%%%%%%%%%%%%%%%%%%%%%%%%%%% enddefinition

%%%%%%%%%%%%%%%%%%%%%%%%%%%%%%%%%%%%%%%%%%%%%%%
%%%%%%%%%%%%%%%%%%%%%%%%%%%%%%%%%%%%%%%%%%%%%%%
%%%%%%%%%%%%%%%%%%%%%%%%%%%%%%%%%%%%%%%%%%%%%%%
% Redefines the bibliography entry for thesis:
%%%%%%%%%%%%%%%%%%%%%%%%%%%%%%%%%%%%%%%%%%%%%%% begindefinition
\DeclareBibliographyDriver{thesis}{%
  \usebibmacro{bibindex}%
  \usebibmacro{begentry}%
  \usebibmacro{author}%
  \setunit{\printdelim{nametitledelim}}\newblock
  \usebibmacro{title}%
  \newunit\newblock
  \printfield{type}%
  \newunit
  \usebibmacro{date}%
  \usebibmacro{finentry}}
%%%%%%%%%%%%%%%%%%%%%%%%%%%%%%%%%%%%%%%%%%%%%%% enddefinition

%%%%%%%%%%%%%%%%%%%%%%%%%%%%%%%%%%%%%%%%%%%%%%%
%%%%%%%%%%%%%%%%%%%%%%%%%%%%%%%%%%%%%%%%%%%%%%%
%%%%%%%%%%%%%%%%%%%%%%%%%%%%%%%%%%%%%%%%%%%%%%%
% Redefines the bibliography entry for unpublished:
%%%%%%%%%%%%%%%%%%%%%%%%%%%%%%%%%%%%%%%%%%%%%%% begindefinition
\DeclareBibliographyDriver{unpublished}{%
  \usebibmacro{bibindex}%
  \usebibmacro{begentry}%
  \usebibmacro{author}%
  \setunit{\printdelim{nametitledelim}}\newblock
  \usebibmacro{title}%
  \newunit\newblock
  \usebibmacro{date}%
  \newunit\newblock
  \iftoggle{bbx:url}
    {\usebibmacro{url+urldate}}
    {}%
  \usebibmacro{finentry}}
%%%%%%%%%%%%%%%%%%%%%%%%%%%%%%%%%%%%%%%%%%%%%%% enddefinition

%%%%%%%%%%%%%%%%%%%%%%%%%%%%%%%%%%%%%%%%%%%%%%%
%%%%%%%%%%%%%%%%%%%%%%%%%%%%%%%%%%%%%%%%%%%%%%%
%%%%%%%%%%%%%%%%%%%%%%%%%%%%%%%%%%%%%%%%%%%%%%%
% Redefines the bibliography field definitions for url and doi to reduce their font size:
%%%%%%%%%%%%%%%%%%%%%%%%%%%%%%%%%%%%%%%%%%%%%%% begindefinition
\DeclareFieldFormat{url}{\mkbibacro{URL}\addcolon\space\footnotesize\url{#1}}
\DeclareFieldFormat{doi}{%
  \mkbibacro{DOI}\addcolon\space\footnotesize
  \ifhyperref
    {\href{https://doi.org/#1}{\nolinkurl{#1}}}
    {\nolinkurl{#1}}}
%%%%%%%%%%%%%%%%%%%%%%%%%%%%%%%%%%%%%%%%%%%%%%% enddefinition

%%%%%%%%%%%%%%%%%%%%%%%%%%%%%%%%%%%%%%%%%%%%%%%
%%%%%%%%%%%%%%%%%%%%%%%%%%%%%%%%%%%%%%%%%%%%%%%
%%%%%%%%%%%%%%%%%%%%%%%%%%%%%%%%%%%%%%%%%%%%%%%
% Allows the positioning of nodes on a circle:
%%%%%%%%%%%%%%%%%%%%%%%%%%%%%%%%%%%%%%%%%%%%%%% begindefinition
\usepackage{tikz}
\usetikzlibrary{chains}
\tikzset{
  nodes around center/.style args={#1:#2:#3:#4}{%
    % #1 = Startwinkel,   #2 = Anzahl Knoten
    % #3 = Zentrums-Node, #4 = Abstand
    at={([shift={(#3)}] {{(\tikzchaincount-1)*360/(#2)+#1}}:{#4})}
  },
  nodes around center*/.style args={#1:#2:#3:#4}{% gleiche Optionen wie oben
    at={([shift={(#3.{(\tikzchaincount-1)*360/(#2)+#1})}] {{(\tikzchaincount-1)*360/(#2)+#1}}:{#4})},
    anchor={(\tikzchaincount-1)*360/(#2)+#1+180}
  }
}
%%%%%%%%%%%%%%%%%%%%%%%%%%%%%%%%%%%%%%%%%%%%%%% enddefinition

%%%%%%%%%%%%%%%%%%%%%%%%%%%%%%%%%%%%%%%%%%%%%%%
%%%%%%%%%%%%%%%%%%%%%%%%%%%%%%%%%%%%%%%%%%%%%%%
%%%%%%%%%%%%%%%%%%%%%%%%%%%%%%%%%%%%%%%%%%%%%%%
% Transforms biblioraphy items:
%%%%%%%%%%%%%%%%%%%%%%%%%%%%%%%%%%%%%%%%%%%%%%% begindefinition
\setbeamertemplate{bibliography item}{%
  \ifboolexpr{ test {\ifentrytype{book}} or test {\ifentrytype{mvbook}}
    or test {\ifentrytype{collection}} or test {\ifentrytype{mvcollection}}
    or test {\ifentrytype{reference}} or test {\ifentrytype{mvreference}} }
    {\setbeamertemplate{bibliography item}[book]}
    {\ifentrytype{online}
       {\setbeamertemplate{bibliography item}[online]}
       {\setbeamertemplate{bibliography item}[article]}}%
  \usebeamertemplate{bibliography item}}
%%%%%%%%%%%%%%%%%%%%%%%%%%%%%%%%%%%%%%%%%%%%%%% enddefinition


\begin{document}

\tikzset{invisible/.style = {opacity = 0},
	    visible on/.style = {alt = {#1{}{invisible}}},
	    alt/.code args = {<#1>#2#3}{\alt<#1>{\pgfkeysalso{#2}}{\pgfkeysalso{#3}}}
}

\pgfplotsset{
	select row/.style = {
    	x filter/.code = {\ifnum\coordindex = #1\else\def\pgfmathresult{}\fi}
	}
}

{\setbeamertemplate{background canvas}{\vbox to \paperheight{\vfil\hbox to \paperwidth{\hfil\begin{adjustbox}{height = .825\paperheight}\TULogo[.2]\end{adjustbox}\hfil}\vfil}}
\begin{frame}
\titlepage
\end{frame}}

\begin{frame}
\frametitle{Table of Contents}
\tableofcontents
\end{frame}

%----------------------------------------------------------------------------------------
%	PRESENTATION SLIDES
%----------------------------------------------------------------------------------------

	\section[Pointer Swizzling in the DBMS Buffer Management]{Pointer Swizzling as in ``In-Memory Performance for Big Data''} \label{sec:paper}

\frame{\sectionpage}

\subsection{Locate Pages in the Buffer Pool without Pointer Swizzling}

\frame{\subsectionpage}

\begin{frame}

	\frametitle{Overview of Search Strategies \large(\cite{Datenbanksysteme_-_Konzepte_und_Techniken_der_Implementierung})}

	\tikzset{%
		every node/.style = {rectangle, draw = black, inner xsep = -0.75mm, very thick, rounded corners}, 
		level 1/.style={level distance = 3cm, sibling distance = 4cm}, 
		level 2/.style={level distance = 3cm, sibling distance = 2cm}, 
		edge from parent/.style = {draw = black, very thick},
		every node/.append style = {draw = black, fill = black!15, text = black}, 
		edge from parent/.append style = {draw = black}
	}

	\centering
	\begin{adjustbox}{width = \textwidth}
		\begin{tikzpicture}
			\node[visible on = <2->]								(algorithms)		[]		{\begin{tabular}{c}Search\\Strategy\end{tabular}}
				child[visible on = <3->] {node[]						(seq)				[]		{\begin{tabular}{c}Direct Search in\\the Buffer Frames\end{tabular}}}
				child[visible on = <4->] {node[]						(aux)				[]		{\begin{tabular}{c}Indirect Search Using\\Auxiliary Tables\end{tabular}}
					child[visible on = <5->] {node[]					(trans)			[]		{\begin{tabular}{c}Translation\\Table\end{tabular}}}
 					child[visible on = <6->] {node[]					(unsort)			[]		{\begin{tabular}{c}Unsorted\\Table\end{tabular}}}
 					child[visible on = <7->] {node[]					(sort)				[]		{\begin{tabular}{c}Sorted\\Table\end{tabular}}}
					child[visible on = <8->] {node[]					(chain)			[]		{\begin{tabular}{c}Chained\\Table\end{tabular}}}
					child[visible on = <9->] {node[]					(tree)			[]		{\begin{tabular}{c}Search\\Trees\end{tabular}}}
					child[visible on = <10->] {node[]					(hash)			[]		{\begin{tabular}{c}Hash\\Table\end{tabular}}}
			};
		
		\end{tikzpicture}
	\end{adjustbox}

\end{frame}

\begin{frame}

	\frametitle{Direct Search in the Buffer Frames \& Unsorted Table}
	
	\begin{block}{\uncover<2->{Direct Search in the Buffer Frames}}
		\begin{itemize}
			\uncover<3->{\item	Checks in each buffer frame the page ID of the contained page}
			\uncover<4->{\item	$T_{\text{avg}}^{\text{search}} \in \mathcal O\left(\frac{n}{2}\right)$, $T_{\text{worst}}^{\text{search}} \in  \mathcal O\left(n\right)$}
			\uncover<5->{\item	The usage of virtual memory management can result in extensive swapping due to read access to many pages!}
		\end{itemize}
	\end{block}

	\begin{block}{\uncover<6->{Unsorted Table}}
		\begin{itemize}
			\uncover<7->{\item	Auxiliary data structure of size $S\text{\tiny{pace}} \in \mathcal O\left(n\right)$}
			\uncover<8->{\item	$T_{\text{avg}}^{\text{search}} \in \mathcal O\left(\frac{n}{2}\right)$, $T_{\text{worst}}^{\text{search}} \in  \mathcal O\left(n\right)$}
		\end{itemize}
	\end{block}

	\tikzset{%
		table/.style = {draw = black, shape = rectangle split, rectangle split parts = 9, rectangle split horizontal, font = \bfseries}
	}

	\centering
	\vspace{-1.0em}
	\uncover<7->{\begin{figure}[ht!]
		\begin{adjustbox}{width = \textwidth}
			\begin{tikzpicture}
				\node[table]	(table)	{\nodepart{one}7785\nodepart{two}6977\nodepart{three}4347\nodepart{four}3380\nodepart{five}5610\nodepart{six}6376\nodepart{seven}4877\nodepart{eight}3332\nodepart{nine}3354};
				
				\foreach \anchor/\address in {one/0, two/1, three/2, four/3, five/4, six/5, seven/6, eight/7, nine/8} {
					\node[]	(\address)		[above = .375cm of table.\anchor]	{\address};
				}
			\end{tikzpicture}
		\end{adjustbox}
		\vspace{-2.0em}
		\caption{An unsorted table used to map buffer frames to page IDs.}
	\end{figure}}

\end{frame}

\begin{frame}

	\frametitle{Translation Table}
	
	\begin{itemize}
		\uncover<2->{\item	Auxiliary data structure with one entry per page in the database $\implies S\text{\tiny{pace}} \in \mathcal O\left(p\right)$}
		\uncover<3->{\item	$T^{\text{search}} \in  \mathcal O\left(1\right)$, $T^{\text{insert}} \in  \mathcal O\left(1\right)$}
	\end{itemize}

	\centering
	\uncover<2->{\begin{figure}[ht!]
		\begin{adjustbox}{width = \textwidth}
			\begin{tabular}{cc}
																		\hline
				\multicolumn{1}{|c|}{0}		& \multicolumn{1}{c|}{$\cdot$} 		\\ 	\hline
				\multicolumn{2}{c}{$\vdots$}                                   		\\ 	\hline
				\multicolumn{1}{|c|}{3331}		& \multicolumn{1}{c|}{$\cdot$} 		\\ 	\hline
				\multicolumn{1}{|c|}{3332}		& \multicolumn{1}{c|}{7}       		\\ 	\hline
				\multicolumn{1}{|c|}{3333}		& \multicolumn{1}{c|}{$\cdot$} 		\\ 	\hline
				\multicolumn{2}{c}{$\vdots$}                                  
			\end{tabular}
			\begin{tabular}{cc}
				\multicolumn{2}{c}{$\vdots$}                                   		\\ 	\hline
				\multicolumn{1}{|c|}{3352}		& \multicolumn{1}{c|}{$\cdot$} 		\\ 	\hline
				\multicolumn{1}{|c|}{3353}		& \multicolumn{1}{c|}{$\cdot$} 		\\ 	\hline
				\multicolumn{1}{|c|}{3354}		& \multicolumn{1}{c|}{8}       		\\ 	\hline
				\multicolumn{1}{|c|}{3355}		& \multicolumn{1}{c|}{$\cdot$} 		\\ 	\hline
				\multicolumn{1}{|c|}{3356}		& \multicolumn{1}{c|}{$\cdot$} 		\\ 	\hline
				\multicolumn{2}{c}{$\vdots$}                                  
			\end{tabular}
			\begin{tabular}{cc}
				\multicolumn{2}{c}{$\vdots$}                                   		\\ 	\hline
				\multicolumn{1}{|c|}{3378}		& \multicolumn{1}{c|}{$\cdot$} 		\\ 	\hline
				\multicolumn{1}{|c|}{3379}		& \multicolumn{1}{c|}{$\cdot$} 		\\ 	\hline
				\multicolumn{1}{|c|}{3380}		& \multicolumn{1}{c|}{3}       		\\ 	\hline
				\multicolumn{1}{|c|}{3381}		& \multicolumn{1}{c|}{$\cdot$} 		\\ 	\hline
				\multicolumn{1}{|c|}{3382}		& \multicolumn{1}{c|}{$\cdot$} 		\\ 	\hline
				\multicolumn{2}{c}{$\vdots$}                                  
			\end{tabular}
			\begin{tabular}{cc}
				\multicolumn{2}{c}{$\vdots$}                                   		\\ 	\hline
				\multicolumn{1}{|c|}{4345}		& \multicolumn{1}{c|}{$\cdot$} 		\\ 	\hline
				\multicolumn{1}{|c|}{4346}		& \multicolumn{1}{c|}{$\cdot$} 		\\ 	\hline
				\multicolumn{1}{|c|}{4347}		& \multicolumn{1}{c|}{2}       		\\ 	\hline
				\multicolumn{1}{|c|}{4348}		& \multicolumn{1}{c|}{$\cdot$} 		\\ 	\hline
				\multicolumn{1}{|c|}{4349}		& \multicolumn{1}{c|}{$\cdot$} 		\\ 	\hline
				\multicolumn{2}{c}{$\vdots$}                                  
			\end{tabular}
			\begin{tabular}{cc}
				\multicolumn{2}{c}{$\vdots$}                                   		\\ 	\hline
				\multicolumn{1}{|c|}{4875}		& \multicolumn{1}{c|}{$\cdot$} 		\\ 	\hline
				\multicolumn{1}{|c|}{4876}		& \multicolumn{1}{c|}{$\cdot$} 		\\ 	\hline
				\multicolumn{1}{|c|}{4877}		& \multicolumn{1}{c|}{6}       		\\ 	\hline
				\multicolumn{1}{|c|}{4878}		& \multicolumn{1}{c|}{$\cdot$} 		\\ 	\hline
				\multicolumn{1}{|c|}{4879}		& \multicolumn{1}{c|}{$\cdot$} 		\\ 	\hline
				\multicolumn{2}{c}{$\vdots$}                                  
			\end{tabular}
			\begin{tabular}{cc}
				\multicolumn{2}{c}{$\vdots$}                                   		\\ 	\hline
				\multicolumn{1}{|c|}{5608}		& \multicolumn{1}{c|}{$\cdot$} 		\\ 	\hline
				\multicolumn{1}{|c|}{5609}		& \multicolumn{1}{c|}{$\cdot$} 		\\ 	\hline
				\multicolumn{1}{|c|}{5610}		& \multicolumn{1}{c|}{4}       		\\ 	\hline
				\multicolumn{1}{|c|}{5611}		& \multicolumn{1}{c|}{$\cdot$} 		\\ 	\hline
				\multicolumn{1}{|c|}{5612}		& \multicolumn{1}{c|}{$\cdot$} 		\\ 	\hline
				\multicolumn{2}{c}{$\vdots$}                                  
			\end{tabular}
			\begin{tabular}{cc}
				\multicolumn{2}{c}{$\vdots$}                                   		\\ 	\hline
				\multicolumn{1}{|c|}{6374}		& \multicolumn{1}{c|}{$\cdot$} 		\\ 	\hline
				\multicolumn{1}{|c|}{6375}		& \multicolumn{1}{c|}{$\cdot$} 		\\ 	\hline
				\multicolumn{1}{|c|}{6376}		& \multicolumn{1}{c|}{5}       		\\ 	\hline
				\multicolumn{1}{|c|}{6377}		& \multicolumn{1}{c|}{$\cdot$} 		\\ 	\hline
				\multicolumn{1}{|c|}{6378}		& \multicolumn{1}{c|}{$\cdot$} 		\\ 	\hline
				\multicolumn{2}{c}{$\vdots$}                                  
			\end{tabular}
			\begin{tabular}{cc}
				\multicolumn{2}{c}{$\vdots$}                                   		\\ 	\hline
				\multicolumn{1}{|c|}{6975}		& \multicolumn{1}{c|}{$\cdot$} 		\\ 	\hline
				\multicolumn{1}{|c|}{6976}		& \multicolumn{1}{c|}{$\cdot$} 		\\ 	\hline
				\multicolumn{1}{|c|}{6977}		& \multicolumn{1}{c|}{1}       		\\ 	\hline
				\multicolumn{1}{|c|}{6978}		& \multicolumn{1}{c|}{$\cdot$} 		\\ 	\hline
				\multicolumn{1}{|c|}{6979}		& \multicolumn{1}{c|}{$\cdot$} 		\\ 	\hline
				\multicolumn{2}{c}{$\vdots$}                                  
			\end{tabular}
			\begin{tabular}{cc}
				\multicolumn{2}{c}{$\vdots$}                                   		\\ 	\hline
				\multicolumn{1}{|c|}{7783}		& \multicolumn{1}{c|}{$\cdot$} 		\\ 	\hline
				\multicolumn{1}{|c|}{7784}		& \multicolumn{1}{c|}{$\cdot$} 		\\ 	\hline
				\multicolumn{1}{|c|}{7785}		& \multicolumn{1}{c|}{0}       		\\ 	\hline
				\multicolumn{1}{|c|}{7786}		& \multicolumn{1}{c|}{$\cdot$} 		\\ 	\hline
				\multicolumn{1}{|c|}{7787}		& \multicolumn{1}{c|}{$\cdot$} 		\\ 	\hline
				\multicolumn{2}{c}{$\vdots$}                                  
			\end{tabular}
		\end{adjustbox}
		\caption{A translation table used to map page IDs to buffer frames.}
	\end{figure}}

\end{frame}

\begin{frame}

	\frametitle{Sorted \& Chained Table}
	
	\begin{block}{\uncover<2->{Sorted Table}}
		\vspace{-1.0em}
		\small
		\begin{itemize}
			\uncover<3->{\item	Auxiliary data structure using a table sorted by page ID only containing cached pages}
			\uncover<4->{\item	$T_{\text{avg}}^{\text{search}} \in  \mathcal O\left(\log_2n\right)$, $T_{\text{avg}}^{\text{insert}} \in  \mathcal O\left(n\log_2n\right)$}
		\end{itemize}
	\end{block}

	\centering
	\vspace{-1.5em}
	\uncover<3->{\begin{figure}[ht!]
		\tikzset{%
			table/.style = {draw = black, shape = rectangle split, rectangle split parts = 9, rectangle split horizontal, font = \bfseries, inner xsep = -2pt, inner ysep = 0pt}
		}

		\begin{adjustbox}{width = .625\textwidth}
			\begin{tikzpicture}
				\node[table]	(table)	{\nodepart{nine}\begin{tabular}{c}7785 \\ $\rightarrow$ 0\end{tabular}
									\nodepart{eight}\begin{tabular}{c}6977 \\ $\rightarrow$ 1\end{tabular}
									\nodepart{four}\begin{tabular}{c}4347 \\ $\rightarrow$ 2\end{tabular}
									\nodepart{three}\begin{tabular}{c}3380 \\ $\rightarrow$ 3\end{tabular}
									\nodepart{six}\begin{tabular}{c}5610 \\ $\rightarrow$ 4\end{tabular}
									\nodepart{seven}\begin{tabular}{c}6376 \\ $\rightarrow$ 5\end{tabular}
									\nodepart{five}\begin{tabular}{c}4877 \\ $\rightarrow$ 6\end{tabular}
									\nodepart{one}\begin{tabular}{c}3332 \\ $\rightarrow$ 7\end{tabular}
									\nodepart{two}\begin{tabular}{c}3354 \\ $\rightarrow$ 8\end{tabular}};
			\end{tikzpicture}
		\end{adjustbox}
		\vspace{-1.0em}
		\caption{A sorted table used to map page IDs to buffer frames.}
	\end{figure}}
	\vspace{-1.5em}

	\begin{block}{\uncover<5->{Chained Table}}
		\vspace{-.5em}
		\small
		\begin{itemize}
			\uncover<6->{\item	Auxiliary data structure using a linked list sorted by page ID only containing cached pages}
			\uncover<7->{\item	$T_{\text{avg}}^{\text{search}} \in  \mathcal O\left(\log_2n\right)$, $T_{\text{avg}}^{\text{insert}} \in  \mathcal O\left(\log_2n\right)$}
			\uncover<8->{\item	Binary search requires more links!}
		\end{itemize}
	\end{block}

	\centering
	\vspace{-1.5em}
	\uncover<6->{\begin{figure}[ht!]
		\tikzset{%
			node distance = .25cm,
			node/.style = {draw = black, shape = rectangle split, rectangle split parts = 2, font = \bfseries, inner xsep = -2pt, inner ysep = 0pt}
		}

		\begin{adjustbox}{width = .625\textwidth}
			\begin{tikzpicture}
				\node[node]	(node0)	[]				{\nodepart{one}\vphantom{M}\nodepart{two}\begin{tabular}{c}3332 \\ $\rightarrow$ 7\end{tabular}};
				\node[node]	(node1)	[right = of node0]	{\nodepart{one}\vphantom{M}\nodepart{two}\begin{tabular}{c}3354 \\ $\rightarrow$ 8\end{tabular}};
				\node[node]	(node2)	[right = of node1]	{\nodepart{one}\vphantom{M}\nodepart{two}\begin{tabular}{c}3380 \\ $\rightarrow$ 3\end{tabular}};
				\node[node]	(node3)	[right = of node2]	{\nodepart{one}\vphantom{M}\nodepart{two}\begin{tabular}{c}4347 \\ $\rightarrow$ 2\end{tabular}};
				\node[node]	(node4)	[right = of node3]	{\nodepart{one}\vphantom{M}\nodepart{two}\begin{tabular}{c}4877 \\ $\rightarrow$ 6\end{tabular}};
				\node[node]	(node5)	[right = of node4]	{\nodepart{one}\vphantom{M}\nodepart{two}\begin{tabular}{c}5610 \\ $\rightarrow$ 4\end{tabular}};
				\node[node]	(node6)	[right = of node5]	{\nodepart{one}\vphantom{M}\nodepart{two}\begin{tabular}{c}6376 \\ $\rightarrow$ 5\end{tabular}};
				\node[node]	(node7)	[right = of node6]	{\nodepart{one}\vphantom{M}\nodepart{two}\begin{tabular}{c}6977 \\ $\rightarrow$ 1\end{tabular}};
				\node[node]	(node8)	[right = of node7]	{\nodepart{one}\vphantom{M}$\cdot$\nodepart{two}\begin{tabular}{c}7785 \\ $\rightarrow$ 0\end{tabular}};
				
				\foreach \this/\next in {0/1, 1/2, 2/3, 3/4, 4/5, 5/6, 6/7, 7/8} {
					\draw[ -> ]		([yshift = 1pt] node\this.mid)	to [bend left = 12.5]		(node\next.text west);
				}
			\end{tikzpicture}
		\end{adjustbox}
		\vspace{-1.0em}
		\caption{A chained table used to map page IDs to buffer frames.}
	\end{figure}}

\end{frame}

\begin{frame}

	\frametitle{Search Trees}

	\small
	\begin{itemize}
		\uncover<2->{\item	Auxiliary data structure is similar to the one of the chained table}
		\uncover<3->{\item	Many different data structures like AVL-trees, red–black trees or splay trees can be used}
		\uncover<4->{\item	$T_{\text{avg}}^{\text{search}} \in  \mathcal O\left(\log n\right)$, $T_{\text{avg}}^{\text{insert}} \in  \mathcal O\left(\log n\right)$}
		\uncover<5->{\item	The worst case costs and the worst cases vary between the different search tree data structures}
	\end{itemize}

	\centering
	\vspace{-1em}
	\uncover<2->{\begin{figure}[ht!]
		\tikzset{%
			every node/.style = {rectangle, draw = black, inner xsep = -0.75mm, inner ysep = -0.25mm, font = \bfseries\Large}, 
			level 1/.style = {level distance = 1.5cm, sibling distance = 6cm}, 
			level 2/.style = {level distance = 1.5cm, sibling distance = 3cm}, 
			level 3/.style = {level distance = 1.5cm, sibling distance = 1.5cm}, 
			edge from parent/.style = {draw = black, ->},
			phantom node/.style = {opacity = 0}
		}

		\centering
		\begin{adjustbox}{totalheight = .4\textheight}
			\begin{tikzpicture}
				\node[]								(4877)		[]		{\begin{tabular}{c}4877 $\rightarrow$ 6\end{tabular}}
					child[] {node[]						(3380)		[]		{\begin{tabular}{c}3380 $\rightarrow$ 3\end{tabular}}
 						child[] {node[]					(3332)		[]		{\begin{tabular}{c}3332 $\rightarrow$ 7\end{tabular}}
  							child[phantom node] {node[]	(phantom0)	[]		{\begin{tabular}{c}3354 $\rightarrow$ 8\end{tabular}}}
  							child[] {node[]				(3354)		[]		{\begin{tabular}{c}3354 $\rightarrow$ 8\end{tabular}}}}
						child[] {node[]					(4347)		[]		{\begin{tabular}{c}4347 $\rightarrow$ 2\end{tabular}}}}
					child[] {node[]						(6376)		[]		{\begin{tabular}{c}6376 $\rightarrow$ 5\end{tabular}}
 						child[] {node[]					(5610)		[]		{\begin{tabular}{c}5610 $\rightarrow$ 4\end{tabular}}}
 						child[] {node[]					(7785)		[]		{\begin{tabular}{c}7785 $\rightarrow$ 0\end{tabular}}
  							child[] {node[]				(6977)		[]		{\begin{tabular}{c}6977 $\rightarrow$ 1\end{tabular}}}
  							child[phantom node] {node[]	(phantom1)	[]		{\begin{tabular}{c}3354 $\rightarrow$ 8\end{tabular}}}}
				};
			
			\end{tikzpicture}
		\end{adjustbox}
		\vspace{-1.0em}
		\caption{A balanced search tree used to map page IDs to buffer frames.}
	\end{figure}}

\end{frame}

\begin{frame}[fragile]

	\frametitle{Hash Table}

	\begin{columns}
      		\column{.5\textwidth}
		
			\tikzset{%
				node distance = .5cm,
				bucketHeaderNumber/.style = {shape = rectangle, draw = black, minimum width = .2\textwidth, anchor = south west, minimum height = .5cm, font = \bfseries},
				bucketHeaderNext/.style = {shape = rectangle, draw = black, minimum width = .2\textwidth, anchor = south east, minimum height = .5cm},
				bucketBody/.style = {shape = rectangle split, rectangle split parts = #1, draw = black, minimum width = .4\textwidth, anchor = north},
				pid/.style = {draw = none, font = \bfseries},
				hashFunction/.style = {draw = black, shape = rectangle, rounded corners = 5pt, minimum height = 1.25\textheight, minimum width = .375\textwidth},
				hashFunctionText/.style = {font = \bfseries},
				hashFunction/.append style = {draw = ForestGreen, fill = ForestGreen!15}, 
				hashFunctionText/.append style = {text = ForestGreen},
				bucketHeaderNumber/.append style = {draw = Purple, fill = Purple!15},
				bucketHeaderNext/.append style = {draw = Purple!75, fill = Purple!11.25}
			}

			\centering
			\begin{adjustbox}{totalheight = \textheight - 4.0em}
				\uncover<2->{\begin{tikzpicture}
					
					% Create the hash function:
					\node[hashFunction]				(func)		[]						{};
					\node[hashFunctionText]			(funcText)		[below = .025cm of func.north]	{$\bm{\text{mod } 5}$};
					
					% Create the hashed Page IDs:
					\pgfmathsetseed{2144}
					\pgfmathsetmacro{\numOfPages}{4}
					\pgfmathsetmacro{\initialValue}{int(10000 * rand)}
					\ifthenelse{\initialValue < 0}{
						\pgfmathsetmacro{\initialValue}{int(-1 * \initialValue)}
					}{}
					\node[pid]							(pid0)		[left = of func]			{\initialValue};
					\node[]							(help0)		[right = of pid0]			{};
					\draw[]		(pid0.east)	--		(help0.west);
					\foreach \x in {1, ..., \numOfPages}{
						\ifthenelse{\x = 1}{
							\pgfmathtruncatemacro{\previousTop}{0}
							\pgfmathtruncatemacro{\previousBottom}{0}
						}{
							\pgfmathtruncatemacro{\previousTop}{(2 * \x) - 3}
							\pgfmathtruncatemacro{\previousBottom}{(2 * \x) - 2}
						}
						\pgfmathtruncatemacro{\top}{(2 * \x) - 1}
						\pgfmathtruncatemacro{\bottom}{(2 * \x)}
						\pgfmathsetmacro{\topValue}{int(10000 * rand)}
						\pgfmathsetmacro{\bottomValue}{int(10000 * rand)}
						\ifthenelse{\topValue < 0}{
							\pgfmathsetmacro{\topValue}{int(-1 * \topValue)}
						}{}
						\ifthenelse{\bottomValue < 0}{
							\pgfmathsetmacro{\bottomValue}{int(-1 * \bottomValue)}
						}{}
						
						\node[pid]							(pid\top)			[above = of pid\previousTop]			{\topValue};
						\node[]							(help\top)			[right = of pid\top]					{};
						\draw[]		(pid\top.east)	--		(help\top.west);
						\node[pid]							(pid\bottom)		[below = of pid\previousBottom]			{\bottomValue};
						\node[]							(help\bottom)		[right = of pid\bottom]				{};
						\draw[]		(pid\bottom)	--		(help\bottom);
					}
					
					% Create the hash buckets:
					\pgfmathsetmacro{\numOfBuckets}{2}
					\pgfmathsetmacro{\initialBucket}{int(\numOfBuckets)}
					\node[bucketBody=2]				(bucketBody2)		[right = of func, yshift = -.25cm]			{\nodepart{one}$4347 \rightarrow 2$\nodepart{two}$4877 \rightarrow 6$};
					\node[bucketBody=2]				(bucketBody1)		[above = 1cm of bucketBody2]			{\nodepart{one}$6376 \rightarrow 5$\nodepart{two}$\cdot$};
					\node[bucketBody=2]				(bucketBody3)		[below = 1cm of bucketBody2]			{\nodepart{one}$\cdot$\nodepart{two}$\cdot$};
					\node[bucketBody=2]				(bucketBody0)		[above = 1cm of bucketBody1]			{\nodepart{one}$3380 \rightarrow 3$\nodepart{two}$5610 \rightarrow 4$};
					\node[bucketBody=2]				(bucketBody4)		[below = 1cm of bucketBody3]			{\nodepart{one}$3354 \rightarrow 8$\nodepart{two}$\cdot$};
					\node[bucketHeaderNumber]		at (bucketBody0.north west)	(bucketHeaderNu0)	[]			{0};
					\node[bucketHeaderNext]			at (bucketBody0.north east)	(bucketHeaderNe0)	[]			{};
					\node[bucketHeaderNumber]		at (bucketBody1.north west)	(bucketHeaderNu1)	[]			{1};
					\node[bucketHeaderNext]			at (bucketBody1.north east)	(bucketHeaderNe1)	[]			{$\cdot$};
					\node[bucketHeaderNumber]		at (bucketBody2.north west)	(bucketHeaderNu2)	[]			{2};
					\node[bucketHeaderNext]			at (bucketBody2.north east)	(bucketHeaderNe2)	[]			{};
					\node[bucketHeaderNumber]		at (bucketBody3.north west)	(bucketHeaderNu3)	[]			{3};
					\node[bucketHeaderNext]			at (bucketBody3.north east)	(bucketHeaderNe3)	[]			{$\cdot$};
					\node[bucketHeaderNumber]		at (bucketBody4.north west)	(bucketHeaderNu4)	[]			{4};
					\node[bucketHeaderNext]			at (bucketBody4.north east)	(bucketHeaderNe4)	[]			{$\cdot$};
					\node[]							(bucketHelp0)		[left = of bucketHeaderNu0]			{};
					\node[]							(bucketHelp1)		[left = of bucketHeaderNu1]			{};
					\node[]							(bucketHelp2)		[left = of bucketHeaderNu2]			{};
					\node[]							(bucketHelp3)		[left = of bucketHeaderNu3]			{};
					\node[]							(bucketHelp4)		[left = of bucketHeaderNu4]			{};
					\draw[]		(bucketHeaderNu0.west)	--		(bucketHelp0.east);
					\draw[]		(bucketHeaderNu1.west)	--		(bucketHelp1.east);
					\draw[]		(bucketHeaderNu2.west)	--		(bucketHelp2.east);
					\draw[]		(bucketHeaderNu4.west)	--		(bucketHelp4.east);

					\node[bucketBody=4]				(bucketBody00)		[right = of bucketBody0, yshift = -.41cm]			{\nodepart{one}$7785 \rightarrow 0$\nodepart{two}$\cdot$\nodepart{three}$\cdot$\nodepart{four}$\cdot$};
					\node[bucketHeaderNumber]		at (bucketBody00.north west)	(bucketHeaderNu00)	[]			{0};
					\node[bucketHeaderNext]			at (bucketBody00.north east)	(bucketHeaderNe00)	[]			{$\cdot$};
					\draw[]		(bucketHeaderNe0.center)	--		(bucketHeaderNu00.west);

					\node[bucketBody=4]				(bucketBody20)		[right = of bucketBody2, yshift = -.445cm]			{\nodepart{one}$3332 \rightarrow 7$\nodepart{two}$6977 \rightarrow 1$\nodepart{three}$\cdot$\nodepart{four}$\cdot$};
					\node[bucketHeaderNumber]		at (bucketBody20.north west)	(bucketHeaderNu20)	[]			{2};
					\node[bucketHeaderNext]			at (bucketBody20.north east)	(bucketHeaderNe20)	[]			{$\cdot$};
					\draw[]		(bucketHeaderNe2.center)	--		(bucketHeaderNu20.west);

					\draw[]		(help5.west)	--		(bucketHelp0.east);
					\draw[]		(help1.west)	--		(bucketHelp0.east);
					\draw[]		(help8.west)	--		(bucketHelp0.east);
					\draw[]		(help4.west)	--		(bucketHelp1.east);
					\draw[]		(help7.west)	--		(bucketHelp2.east);
					\draw[]		(help3.west)	--		(bucketHelp2.east);
					\draw[]		(help0.west)	--		(bucketHelp2.east);
					\draw[]		(help6.west)	--		(bucketHelp2.east);
					\draw[]		(help2.west)	--		(bucketHelp4.east);
				\end{tikzpicture}}
			\end{adjustbox}
		
		\column{.5\textwidth}
			\begin{itemize}
				\uncover<3->{\item	Each page ID is mapped to a hash bucket using a hash function}
				\uncover<4->{\item	Only the page IDs of buffered pages are in the hash table}
				\uncover<5->{\item	If a hash bucket is full, a chained bucket gets added}
				\uncover<6->{\item	$T_{\text{avg}}^{\text{search}} \in  \mathcal O\left(1\right)$, \\$T_{\text{avg}}^{\text{insert}} \in  \mathcal O\left(1\right)$, \\$T_{\text{worst}}^{\text{search}} \in  \mathcal O\left(n\right)$}
			\end{itemize}
				
	\end{columns}

\end{frame}

\begin{frame}[fragile]

	\frametitle{Locate Pages in Buffer Pool with Hash Table \large(\cite{Graefe:2014})}
	
	\tikzset{%
		node distance = 1cm,
		startstop/.style = {rectangle, rounded corners, minimum width = 2cm, minimum height = 1cm, draw = red, fill = red!15, text width = 3cm, align = center},
		io/.style = {trapezium, trapezium left angle = 70, trapezium right angle = 110, minimum width = 2cm, minimum height = 1cm, draw = blue, fill = blue!15, text width = 2cm, align = center},
		process/.style = {rectangle, minimum width = 2cm, minimum height = 1cm, draw = orange, fill = orange!15, text width = 3cm, align = center},
		decision/.style = {diamond, aspect=2, minimum width = 2cm, minimum height = 1cm, draw = green, fill = green!15, text width = 3cm, align = center},
		arrow/.style = { -> , thick, >=stealth}
	}

	\noindent\makebox[\textwidth]{
		\begin{adjustbox}{width = \textwidth + 4.0em}
			\begin{tikzpicture}
				\node[visible on = <1->]							(center start)		[]						{};
				\node[visible on = <2->, io]						(page)			[left = of center start]			{Buffer pool page image};
				\node[visible on = <3->, io]						(key)				[right = of center start]		{Search key};
				\node[visible on = <4->, process]					(search)			[below = of center start]		{Look for entry in page image that corresponds to search key};
				\node[visible on = <5->, decision]					(found1)			[below = of search]			{Found entry?};
				\node[visible on = <6->, startstop]					(miss)			[below = of found1]			{Search key not found};
				\node[visible on = <7->, process]					(get id)			[right = of found1]			{Get identifier of the next page to search from the page image};
				\node[visible on = <8->, process]					(hash)			[below = of get id]			{calculate hash id of the page id};
				\node[visible on = <9->, process]					(lookup)			[right = 1.5cm of hash]		{Look in buffer pool hash table for hashed page id (protect hash table)};
				\node[visible on = <10->, decision]					(found2)			[above = of lookup]			{Found hashed page id?};
				\node[visible on = <12->, process]					(load)			[above = of found2]			{Bring page into buffer pool (possibly need to evict another page image)};
				\node[visible on = <11->, startstop]					(final)			[right = of found2]			{Return buffer pool page image of the next page to search};

				\draw[visible on = <4->, arrow]	(page)			-|						(search);
				\draw[visible on = <4->, arrow]	(key)				-|						(search);
				\draw[visible on = <5->, arrow]	(search)			--						(found1);
				\draw[visible on = <6->, arrow]	(found1)			--	node[right]	{no}		(miss);
				\draw[visible on = <7->, arrow]	(found1)			--	node[above]	{yes}		(get id);
				\draw[visible on = <8->, arrow]	(get id)			--						(hash);
				\draw[visible on = <9->, arrow]	(hash)			--						(lookup);
				\draw[visible on = <10->, arrow]	(lookup)			--						(found2);
				\draw[visible on = <12->, arrow]	(found2)			--	node[right]	{no}		(load);
				\draw[visible on = <11->, arrow]	(found2)			--	node[above]	{yes}		(final);
				\draw[visible on = <13->, arrow]	(load)			-|						(final);
			\end{tikzpicture}
		\end{adjustbox}
	}

\end{frame}

\subsection{Locate Pages in the Buffer Pool with Pointer Swizzling}

\frame{\subsectionpage}

\begin{frame}

	\frametitle{Pointer Swizzling}
	
	\begin{block}{Definition}
		\justify
		To swizzle a pointer means to transform the address of the persistent object referenced there to a more direct address of the transient object in a way that this transformation could be used during multiple indirections of this pointer (\cite{Moss:1992}).
	\end{block}
	
\end{frame}

\begin{frame}[fragile]

	\frametitle{Classification of the Pointer Swizzling Approach following \cite{White:1995}}

	\tikzset{%
		selected/.style = {font = \bfseries, very thick},
		selected/.append style = {text = blue, color = blue}
	}

	\vspace{-1.5em}
	\centering
	\begin{adjustbox}{totalheight = \textheight - 2.0em}
		\begin{tikzpicture}
			\draw arc [start angle = 0,   
                 				end angle = 360,
                					x radius = 4cm, 
                 				y radius = 4cm]
						node [visible on = <5->, pos = 1 * 1/14]				(eager)		{eager}
						node [visible on = <6->, pos = 2 * 1/14, selected]		(direct)		{direct}
						node [visible on = <7->, pos = 3 * 1/14, selected]		(in-place)		{in-place}
						node [visible on = <3->, pos = 4 * 1/14]				(hardware)	{hardware}
						node [visible on = <2->, pos = 5 * 1/14]				(no-swizzling)	{no-swizzling}
						node [visible on = <8->, pos = 6 * 1/14]				(no uncaching)	{no uncaching}
						node [visible on = <4->, pos = 7 * 1/14]				(partial)		{partial}
						node [visible on = <5->, pos = 8 * 1/14, selected]		(lazy)		{lazy}
						node [visible on = <6->, pos = 9 * 1/14]				(indirect)		{indirect}
						node [visible on = <7->, pos = 10 * 1/14]				(copy)		{copy}
						node [visible on = <3->, pos = 11 * 1/14, selected]		(software)		{software}
						node [visible on = <2->, pos = 12 * 1/14, selected]		(swizzling)		{swizzling}
						node [visible on = <8->, pos = 13 * 1/14, selected]		(uncaching)	{uncaching}
						node [visible on = <4->, pos = 14 * 1/14, selected]		(full)			{full};
			
			\node[anchor = center, right = 4cm of partial.center]			(center)			{};
			\path[->]	
				(center.center)		edge[visible on = <6->, selected]	(direct)
				(center.center)		edge[visible on = <6->]			(indirect)
				(center.center)		edge[visible on = <7->, selected]	(in-place)
				(center.center)		edge[visible on = <7->]			(copy)
				(center.center)		edge[visible on = <3->]			(hardware)
				(center.center)		edge[visible on = <3->, selected]	(software)
				(center.center)		edge[visible on = <4->]			(partial)
				(center.center)		edge[visible on = <4->, selected]	(full)
				(center.center)		edge[visible on = <2->]			(no-swizzling)
				(center.center)		edge[visible on = <2->, selected]	(swizzling)
				(center.center)		edge[visible on = <5->]			(eager)
				(center.center)		edge[visible on = <5->, selected]	(lazy)
				(center.center)		edge[visible on = <8->]			(no uncaching)
				(center.center)		edge[visible on = <8->, selected]	(uncaching);
		\end{tikzpicture}
	\end{adjustbox}

\end{frame}

\begin{frame}[fragile]

	\frametitle{Locate Pages in Buffer Pool w/ Pointer Swizzling \large(\cite{Graefe:2014})}

	\tikzset{%
		node distance = 1cm,
		startstop/.style = {rectangle, rounded corners, minimum width = 3cm, minimum height = 1cm,text centered, draw = red, fill = red!15},
		io/.style = {trapezium, trapezium left angle = 70, trapezium right angle = 110, minimum width = 3cm, minimum height = 1cm, text centered, draw = blue, fill = blue!15},
		process/.style = {rectangle, minimum width = 3cm, minimum height = 1cm, text centered, draw = orange, fill = orange!15},
		decision/.style = {diamond, minimum width = 3cm, minimum height = 1cm, text centered, draw = green, fill = green!15},
		arrow/.style = { -> , thick, >=stealth}
	}

	\noindent\makebox[\textwidth]{
		\begin{adjustbox}{width = \textwidth + 4.0em}
			\begin{tikzpicture}
				\node[visible on = <1->]							(center start)		[]						{};
				\node[visible on = <2->, io]						(page)			[left = of center start]			{\begin{tabular}{c}Buffer pool\\page image\end{tabular}};
				\node[visible on = <3->, io]						(key)				[right = of center start]		{Search key};
				\node[visible on = <4->, process]					(search)			[below = of center start]		{\begin{tabular}{c}Look for entry\\in page image\\that corresponds\\to search key\end{tabular}};
				\node[visible on = <5->, decision]					(found)			[below = of search]			{\begin{tabular}{c}Found\\entry?\end{tabular}};
				\node[visible on = <6->, startstop]					(miss)			[below = of found]			{\begin{tabular}{c}Search key\\not found\end{tabular}};
				\node[visible on = <7->, process]					(get id)			[right = of found]			{\begin{tabular}{c}Get identifier of\\the next page\\to search from\\the page image\end{tabular}};
				\node[visible on = <8->, decision]					(swizzled)			[right = of get id]			{\begin{tabular}{c}Identifier\\swizzled?\end{tabular}};
				\node[visible on = <10->, process]					(load)			[above = of swizzled]			{\begin{tabular}{c}Bring page into\\buffer pool (possibly\\need to evict another\\ page image) and\\swizzle pointer on it\end{tabular}};
				\node[visible on = <9->, startstop]					(final)			[right = of swizzled]			{\begin{tabular}{c}Return buffer pool\\page image of the next\\page to search\end{tabular}};

				\draw[visible on = <4->, arrow]	(page)			-|						(search);
				\draw[visible on = <4->, arrow]	(key)				-|						(search);
				\draw[visible on = <5->, arrow]	(search)			--						(found);
				\draw[visible on = <6->, arrow]	(found)			--	node[right]	{no}		(miss);
				\draw[visible on = <7->, arrow]	(found)			--	node[above]	{yes}		(get id);
				\draw[visible on = <8->, arrow]	(get id)			--						(swizzled);
				\draw[visible on = <10->, arrow]	(swizzled)			--	node[right]	{no}		(load);
				\draw[visible on = <9->, arrow]	(swizzled)			--	node[above]	{yes}		(final);
				\draw[visible on = <11->, arrow]	(load)			-|						(final);
			\end{tikzpicture}
		\end{adjustbox}
	}

\end{frame}

	
	\section[Performance Evaluation of Pointer Swizzling]{Performance Evaluation of the Buffer Management Utilizing Pointer Swizzling} \label{sec:performance}

\frame{\sectionpage}

\subsection{Expected Performance}

\frame{\subsectionpage}

\begin{frame}[fragile]

	\frametitle{Performance of Different Buffer Pool Sizes}
	
	\vspace{1.0em}

	\tikzset{%
		plot/.style = {smooth, mark = \empty},
		optimal/.style = {very thick},
		best/.style = {},
		random/.style = {very thick},
		possible/.style = {pattern = north east lines},
		optimal/.append style = {color = green},
		best/.append style = {color = blue},
		random/.append style = {color = red},
		possible/.append style = {pattern color = blue!75}
	}
		
	\centering
	\begin{adjustbox}{width = \textwidth}
		\begin{tikzpicture}
			\begin{axis}[xlabel = \large buffer pool size $B$,
					   xmin = 0,
					   xmax = 105,
					   xtick = {5, 100},
					   xticklabels = {$B_{min}$, $D$},
					   ylabel = \large miss rate $MR$,
					   ymin = 0,
					   ymax = 105,
					   ytick = {5, 95, 100},
					   yticklabels = {$MR_{CS}$, $MR_{max}$, $1$},
					   legend pos = north east,
					   area legend,
					   axis x line = bottom,
					   axis y line = left,
					   width = \textwidth,
					   height = {\textheight - 2.0em}]
				\addplot+[visible on = <2->, plot, optimal, name path = optimal, line legend] coordinates {
					(5, 95)
					(20, 25)
					(50, 7.5)
					(100, 5)
					};
				\addplot+[visible on = <4->, plot, best, name path = best, empty legend] coordinates {
					(5, 95)
					(20, 45)
					(30, 27.5)
					(50, 15)
					(100, 5)
					};
				\addplot+[visible on = <3->, plot, random, name path = random, line legend] coordinates {
					(5, 95)
					(20, 60)
					(35, 34)
					(50, 22.5)
					(100, 5)
					};
							
				\addplot[visible on = <4->, possible] fill between [of = best and random];
				
				\legend{optimal, , random, possible}
			\end{axis}
		\end{tikzpicture}
	\end{adjustbox}
	\vspace{-3.0em}
	\begin{flushright}\cite{Effelsberg:1984}\cite{Datenbanksysteme_-_Konzepte_und_Techniken_der_Implementierung}\end{flushright}
	
\end{frame}

\begin{frame}[fragile]

	\frametitle{Buffer Management with and without Pointer Swizzling
}
	
	\vspace{1.0em}

	\tikzset{%
		plot/.style = {smooth, mark = \empty},
		swizzling/.style = {plot},
		noswizzling/.style = {plot},
		threshold/.style = {draw = black},
		thresholdArrow/.style = {draw = black, ->},
		thresholdNode/.style = {draw = none, font = \tiny, shape = rectangle, anchor = north, yshift = -5, text width = 45, align = center},
		swizzling/.append style = {color = green, fill opacity = 0.75},
		noswizzling/.append style = {color = red, fill opacity = 0.75},
		swizzlingHit/.style = {swizzling, fill = green!60},
		swizzlingMiss/.style = {swizzling, fill = green!30},
		noswizzlingHit/.style = {noswizzling, fill = red!60},
		noswizzlingMiss/.style = {noswizzling, fill = red!30},
		threshold/.append style = {thick},
		thresholdArrow/.append style = {thick},
		thresholdNode/.append style = {fill = none}
	}
	
	\centering
	\begin{adjustbox}{width = \textwidth}
		\begin{tikzpicture}
			\begin{axis}[xlabel = \large hit rate,
					   xmin = 0,
					   xmax = 100,
					   xtick = {5, 95},
					   xticklabels = {$HR_{min}$, $HR_{CS}$},
					   ylabel = \large total execution time,
						   y label style = {at = {(axis description cs:0.125, 0.5)}, anchor = south},
					   ylabel near ticks,
					   ymin = 0,
					   ymax = 101,
					   ytick = \empty,
					   stack plots = y,
					   axis x line = bottom,
					   axis y line = left,
					   width = \textwidth,
					   height = {\textheight - 2.0em}]
				\addplot[noswizzlingHit, visible on = <2->, name path = noswizzlingHit] coordinates {
					(5, 3)
					(95, 60)}
					\closedcycle; \label{noswizzlingHits}
				\addplot[noswizzlingMiss, visible on = <4->, name path = noswizzlingMiss] coordinates {
					(5, 80)
					(95, 4)}
					\closedcycle; \label{noswizzlingMisses}
			\end{axis}
			
			\begin{axis}[xlabel = \large hit rate,
					   xmin = 0,
					   xmax = 100,
					   xtick = {5, 95},
					   xticklabels = {$HR_{min}$, $HR_{CS}$},
					   ylabel = \large total execution time,
					   ymin = 0,
					   ymax = 101,
					   ytick = \empty,
					   stack plots = y,
				  	   legend style = {at = {(1.05, 1.05)}, font = \tiny},
					   area legend,
					   axis x line = bottom,
					   axis y line = left,
					   width = \textwidth,
					   height = {\textheight - 2.0em},
					   hide axis]
				\addlegendimage{/pgfplots/refstyle = noswizzlingHits}
				\addlegendentry{Page Hits without Pointer Swizzling}
				\addlegendimage{/pgfplots/refstyle = noswizzlingMisses}
				\addlegendentry{Page Misses without Pointer Swizzling}
				
				\path[visible on = <6->, threshold]			(33.42, 0)			--		(33.42, 101);
				\path[visible on = <6->, thresholdArrow]	(33.42, 62.5)		--	node[thresholdNode, xshift = -15]	{faster without pointer swizzling}	(33.42 - 10, 62.5);
				\path[visible on = <6->, thresholdArrow]	(33.42, 62.5)		--	node[thresholdNode, xshift = 15]	{faster with pointer swizzling}	(33.42 + 10, 62.5);
				\addplot[swizzlingHit, visible on = <3->, name path = swizzlingHit] coordinates {
					(5, 1)
					(95, 20)}
					\closedcycle;
					\addlegendentry{Page Hits with Pointer Swizzling}
				\addplot[swizzlingMiss, visible on = <5->, name path = swizzlingMiss] coordinates {
					(5, 100)
					(95, 5)}
					\closedcycle;
					\addlegendentry{Page Misses with Pointer Swizzling}
			\end{axis}
		\end{tikzpicture}
	\end{adjustbox}
	
\end{frame}

\subsection{Measured Performance}

\frame{\subsectionpage}

\begin{frame}[fragile]

	\frametitle{Transaction Throughput (TPC-C)}
	
	\tikzset{%
		DBSize/.style = { - , thick, dotted},
		DBSizeMark/.style = {draw = none, rotate = 90, anchor = south},
		traditionalStyle/.style = {color = blue},
		swizzlingStyle/.style = {color = black},
		DBSize/.append style = {draw = purple},
		DBSizeMark/.append style = {text = purple}
	}

	\centering
	\begin{adjustbox}{totalheight = \textheight - 2.0em}
		\begin{tikzpicture}[spy using outlines =  {square, magnification = 3, connect spies}]
			\begin{axis}[xlabel = {\ttfamily sm\_bufpoolsize $\left[\si{\gibi\byte}\right]$},
					   xlabel near ticks,
					   xmin = 0,
					   xmax = 20000,
					   xtick distance = {1000},
					   xticklabels = {,0,1,...,20},
					   scaled x ticks = false,
					   minor x tick num = 9,
					   ylabel = {$\text{average transaction throughput }\left[\sfrac{\text{transactions}}{\text{s}}\right]$},
					   ylabel near ticks,
					   ymin = 0,
					   ymax = 18000,
					   ymode = normal,
					   scaled y ticks = false,
					   grid = major,
					   legend entries = {Traditional Buffer Pool, Pointer Swizzling Buffer Pool},
					   legend pos = south east,
					   width = 1.3\textwidth,
					   height = 1.25\textheight]		
				\draw[DBSize]				(axis cs: 13215, 0)			edge		(axis cs: 13215, 18000);
				\node[DBSizeMark]			at (axis cs: 13215, 9000)		{initial database size};

				\addplot[visible on = <2->, traditionalStyle, error bars/.cd, y dir = both, y explicit] table[x = buffersize, y = averagequerythroughput, y error = standarddeviationquerythroughput] {./tex/data/tpcc_noswizzling_performance.csv};
				\addplot[visible on = <3->, swizzlingStyle, error bars/.cd, y dir = both, y explicit] table[x = buffersize, y = averagequerythroughput, y error = standarddeviationquerythroughput] {./tex/data/tpcc_swizzling_performance.csv};
			\end{axis}
		\end{tikzpicture}
	\end{adjustbox}
	
\end{frame}

\begin{frame}[fragile]

	\frametitle{Transaction Throughput (TPC-B)}
	
	\tikzset{%
		DBSize/.style = { - , thick, dotted},
		DBSizeMark/.style = {draw = none, rotate = 90, anchor = south},
		traditionalStyle/.style = {color = blue},
		swizzlingStyle/.style = {color = black},
		DBSize/.append style = {draw = purple},
		DBSizeMark/.append style = {text = purple}
	}

	\centering
	\begin{adjustbox}{totalheight = \textheight - 3.0em}
		\begin{tikzpicture}[spy using outlines =  {square, magnification = 3, connect spies}]
			\begin{axis}[xlabel = {\ttfamily sm\_bufpoolsize $\left[\si{\gibi\byte}\right]$},
					   xlabel near ticks,
					   xmin = 0,
					   xmax = 5000,
					   xtick distance = {1000},
					   xticklabels = {,0,1,...,5},
					   scaled x ticks = false,
					   minor x tick num = 9,
					   ylabel = {$\text{average transaction throughput }\left[\sfrac{\text{transactions}}{\text{s}}\right]$},
					   ylabel near ticks,
					   ymin = 0,
					   ymax = 55000,
					   ymode = normal,
					   scaled y ticks = false,
					   grid = major,
					   legend entries = {Traditional Buffer Pool, Pointer Swizzling Buffer Pool},
					   legend pos = south east,
					   width = 1.35\textwidth,
					   height = 1.25\textheight]		
				\draw[DBSize]				(axis cs: 1870, 0)			edge		(axis cs: 1870, 55000);
				\node[DBSizeMark]			at (axis cs: 1870, 17500)		{initial database size};

				\addplot[visible on = <2->, traditionalStyle, error bars/.cd, y dir = both, y explicit] table[x = buffersize, y = averagequerythroughput, y error = standarddeviationquerythroughput] {./tex/data/tpcb_noswizzling_performance.csv};
				\addplot[visible on = <3->, swizzlingStyle, error bars/.cd, y dir = both, y explicit] table[x = buffersize, y = averagequerythroughput, y error = standarddeviationquerythroughput] {./tex/data/tpcb_swizzling_performance.csv};
			\end{axis}
		\end{tikzpicture}
	\end{adjustbox}
	
\end{frame}

\begin{frame}[fragile]

	\frametitle{Buffer Pool Performance Acquiring Shared Latches}
	
	\pgfplotsset{%
		every axis/.append style = {
			ylabel = {$\text{execution time }\left[\si{\nano\second}\right]$},
			ylabel near ticks,
			y label style = {font = \small},
			ymin = 1,
			yticklabel style = {font = \tiny},
			minor y tick num = 9,
			ymode = log,
			scaled y ticks = false,
			ybar,
			xmin = -2,
			xmax = 4,
			bar width = .6em,
			xtick = {-1, 0, 1, 2, 3},
			xticklabels = {{total}, {hit}, {miss}, {miss\\w/o\\evict}, {miss\\w/\\evict}},
			x tick label style = {align = center, font = \footnotesize},
			xlabel near ticks,
			ymajorgrids = true,
			width = .6\textwidth,
			height = {\textheight - 2.0em}
		}
	}

	\tikzset{%
		noSwizzle/.style = {very thick},
		swizzle/.style = {very thick},
		noSwizzle/.append style = {draw = blue, fill = blue!50},
		swizzle/.append style = {draw = black, fill = black!25}
	}

	\newcommand{\noeviction}{
		\begin{tikzpicture}
			\begin{axis}[xlabel = {\SI{20}{\gibi\byte} (w/o eviction)},
					  legend entries = {Traditional Buffer Pool, Pointer Swizzling Buffer Pool},
					  legend to name = legendname,
					  legend style = {font = \footnotesize, legend columns = -1, /tikz/every even column/.append style = {column sep = 0.5cm}}]
				\addplot[noSwizzle, visible on = <2->] coordinates
					{(-1, 1268.48) (0, 1132.46) (1, 59370.2) (2, 59370.2) (3, 0)};
				\addplot[swizzle, visible on = <3->] coordinates
					{(-1, 1277.03) (0, 771.499) (1, 212478) (2, 212478) (3, 0)};
			\end{axis}
		\end{tikzpicture}
	}

	\newcommand{\eviction}{
		\begin{tikzpicture}
			\begin{axis}[xlabel = {\SI{500}{\mebi\byte} (w/ eviction)}]
				\addplot[noSwizzle, visible on = <4->] coordinates
					{(-1, 299457) (0, 262233) (1, 1.73524e+06) (2, 22653.6) (3, 2.24109e+09)};
				\addplot[swizzle, visible on = <5->] coordinates
					{(-1, 343323) (0, 299218) (1, 2.3168e+06) (2, 200154) (3, 1.98806e+09)};
			\end{axis}
		\end{tikzpicture}
	}

	\vspace{1.0em}

	\centering
	\begin{adjustbox}{totalheight = \textheight - 3.0em}
		\begin{tabular}{cc}
			\multicolumn{2}{c}{\ref{legendname}}					\\
			\scalebox{1}{\eviction}	&	\scalebox{1}{\noeviction}
		\end{tabular}
	\end{adjustbox}
	
\end{frame}

\subsection{Conclusion}

\frame{\subsectionpage}

\begin{frame}

	\frametitle{Conclusion}
	
	\begin{block}{\uncover<2->{Overall Performance}}
		\begin{itemize}
			\uncover<3->{\item	Pointer swizzling couldn't improve the performance on TPC-C benchmark runs with a duration of \SI{10}{\minute}.}
			\uncover<4->{\item	The page hits after the cold start couldn't compensate the overhead of pointer swizzling during the cold start.}
			\uncover<5->{\item	A continuously running DB with large buffer pool could profit from pointer swizzling.}
		\end{itemize}
	\end{block}
	
	\begin{block}{\uncover<6->{Buffer Pool Performance}}
		\begin{itemize}
			\uncover<7->{\item	A page hit is faster when pointer swizzling is activated.}
			\uncover<8->{\item	A page miss is slower when pointer swizzling is activated.}
			\uncover<9->{\item	After the cold start phase, activated pointer swizzling will improve the buffer pool performance for large buffer pools.}
		\end{itemize}
	\end{block}

\end{frame}


	\section[Page Eviction Strategies]{Page Eviction Strategies in the Context of Pointer Swizzling} \label{sec:eviction}

\frame{\sectionpage}

\begin{frame}

	\frametitle{Motivation \uline{not} to Analyze Different Page Eviction Strategies}
	
	\begin{itemize}
		\uncover<2->{\item	Even LRU results in decent hit rates}
	\end{itemize}
	
	\tikzset{%
		every node/.style = {font = \sffamily},
		phantom/.style = {rectangle, draw = none, thick},
		layer/.style = {rectangle, draw, thick},
		toplayers/.style = {layer},
		buffer/.style = {layer},
		storage/.style = {layer},		
		disk/.style = {cylinder, cylinder uses custom fill, shape border rotate = 90, draw, minimum height = 1cm, minimum width = 1.5cm, thick},
		disktext/.style = {},
		interfacearrow/.style = {<->, thick, draw = black}
	}

	\centering
	\uncover<2->{\begin{adjustbox}{width = \textwidth}
		\begin{tikzpicture}
			\begin{axis}[title = {TPC-C with Warehouses: 100, Threads: 25},
					   axis on top,
			 		   width = \textwidth,
					   height = {\textheight - 6em},
					   xlabel = {LRU stack depth},
					   xlabel near ticks,
					   xmode = log,
					   xmin = 5e-1,
					   xmax = 1e8,
					   ymode = normal,
					   ybar interval,
					   x tick label as interval = false,
					   xtick = {},
					   xtickten = {0, 1, ..., 8},
					   scaled y ticks = false,
					   ylabel = {\# of references},
					   ylabel near ticks,
					   grid = none]
				\addplot [fill = gray!50] table [x = Lower, y = Count] {./tex/data/lru_stackdepth_histograms/tpcc/ln_histogram_lrused_stackdepth/ln_histogram_LRUsed_stackdepth_analysis_wh100_t25.csv};
			\end{axis}
		\end{tikzpicture}
	\end{adjustbox}}
	
\end{frame}

\begin{frame}

	\frametitle{But ...}
	
	\begin{itemize}
		\uncover<2->{\item	Page reference pattern containing a loop slightly to long to fit in the buffer pool:}
			\begin{itemize}
				\uncover<3->{\item	\textbf{OPT:} Hit rate close to \num{1}}
				\uncover<4->{\item	\textbf{LRU:} Hit rate of \num{0}}
			\end{itemize}
		\uncover<5->{\item	Some pages gets referenced very frequent for a limited time:}
			\begin{itemize}
				\uncover<6->{\item	\textbf{OPT:} Pages would be evicted after their last reference}
				\uncover<7->{\item	\textbf{LFU:} Pages waste buffer frames probably during the whole running time of the DB}
			\end{itemize}
		\uncover<8->{\item	Huge access time gap $\implies$ Every saved page miss significantly improves the performance}
		\uncover<9->{\item	Pointer swizzling even amplifies that effect}
	\end{itemize}
		
\end{frame}

\subsection{Probable pitfalls when Implementing a Page Eviction Strategy for a DBMS Buffer Manager}

\frame{\subsectionpage}

\begin{frame}

	\frametitle{General Problems Concerning DBMS Buffer Managers}

	\begin{itemize}
		\uncover<2->{\item	Fixed pages cannot be evicted but a long timespan between a fix and an unfix of a page could make it a candidate for eviction.}
		\uncover<3->{\item	A page pinned for refix cannot be evicted but a long timespan in which a page is pinned could make it a candidate for eviction.}
		\uncover<4->{\item	Dirty pages cannot be evicted but a page being dirty for a long timespan due to the update propagation using write-back policy could make it a candidate for eviction.}
	\end{itemize}

\end{frame}

\begin{frame}

	\frametitle{Additional Problem When Using Pointer Swizzling}

	\begin{itemize}
		\uncover<2->{\item	A page containing swizzled pointer cannot be evicted but a page unfixed before the last unfix of one of its child pages could make it a candidate for eviction before its child pages got evicted.}
	\end{itemize}

\end{frame}

\begin{frame}

	\frametitle{Solutions}

	\begin{itemize}
		\uncover<2->{\item	Check each of the restrictions before the eviction of a page.}
		\uncover<3->{\item	Update the statistics of the eviction strategy during an unfix, too.}
		\uncover<4->{\item	Update the statistics of the eviction strategy during an pin and unpin, too.}
		\uncover<5->{\item	Use write-thru for update propagation or a page cleaner decoupled from the buffer pool as proposed in \cite{Sauer:2016}.}
		\uncover<6->{\item	Use a page eviction strategy that takes into account the content of pages (like the structure of an B tree).}
	\end{itemize}

\end{frame}

\subsection{Evaluated Page Replacement Strategies}

\frame{\subsectionpage}

\begin{frame}

	\frametitle{RANDOM}
	
	\begin{block}{\uncover<1->{Overview}}
		\begin{itemize}
			\uncover<2->{\item	Simplest page eviction strategy}
			\uncover<3->{\item	Evicts a random page that can be evicted}
			\uncover<4->{\item	Won't evict frequently used pages as they're latched all the time}
		\end{itemize}
	\end{block}

\end{frame}

\begin{frame}

	\frametitle{GCLOCK}
	
	\begin{block}{\uncover<1->{Overview}}
		\begin{itemize}
			\uncover<2->{\item	Slight enhancement of the CLOCK algorithm: \emph{generalized CLOCK}}
			\uncover<3->{\item	Uses finer-grained statistics about the recency of page references}
			\uncover<4->{\item	Parameter $k$ defines granulation of statistics}
			\begin{itemize}
				\uncover<5->{\item	\textbf{$\bm{k = 1}$: }CLOCK}
				\uncover<6->{\item	\textbf{$\bm{k = \#\text{frames}}$: }Similar to LRU}
			\end{itemize}
		\end{itemize}
	\end{block}

\end{frame}

\begin{frame}[fragile]

	\frametitle{GCLOCK}
	
	\tikzset{%
		node distance = 1cm,
		refBit/.style = {draw = black, shape = rectangle},
		hand/.style = {very thick, draw = black},
		stack/.style = {rectangle split, rectangle split parts = #1, draw = black}
	}

	\begin{block}{Example}
		\centering
		\begin{adjustbox}{totalheight = \textheight - 5.0em}
			\begin{tikzpicture}
				\node[]		(t1_center)	[]						{};
				\begin{scope}[start chain = circle placed {nodes around center = 45:24:t1_center:7.5em},
						      every node/.append style = {on chain = circle}]
					\foreach \cnt/\text in {0/4, 1/3, 2/2, 3/4, 4/4, 5/4, 6/0, 7/1, 8/1, 9/0, 10/0, 11/0, 12/0, 13/3, 14/3, 15/3, 16/2, 17/2, 18/1, 19/4, 20/4, 21/0, 22/1, 23/0}
						\node[refBit] (t1_\cnt) {\text};
					\chainin (circle-begin);
					\path[->]	
						(t1_center.center)		edge[hand]		(t1_12);
					\path[->]	
						($(t1_center.center)!0.8!(t1_12)$)		edge[bend left = 15]		($(t1_center.center)!0.8!(t1_10)$);
					\node[]		(t1_head)	[left = -.125em of t1_12]	{\small head};
					\node[]		(t1_tail)	[left = -.125em of t1_13]	{\small tail};
				\end{scope}
			\end{tikzpicture}
		\end{adjustbox}
	\end{block}

\end{frame}

\begin{frame}

	\frametitle{GCLOCK}
	
	\begin{block}{\uncover<1->{Advantage of Higher $k$-Values}}
		\begin{itemize}
			\uncover<3->{\item	More detailed statistics about page references \\ $\implies$ Higher hit rate \\ $\implies$ Higher performance}
		\end{itemize}
	\end{block}
	\begin{block}{\uncover<2->{Advantages of Lower $k$-Values}}
		\begin{itemize}
			\uncover<4->{\item	Lower processing time required to find an eviction victim \\ $\implies$ Higher performance}
			\uncover<5->{\item	Lower memory overhead due to shorter \lstinline{referenced}-numbers}
		\end{itemize}
	\end{block}
	
	\uncover<6->{$\implies$ Trade-off between CPU- and I/O-optimization}

\end{frame}

\begin{frame}

	\frametitle{CAR}
	
	\begin{block}{\uncover<1->{Overview}}
		\begin{itemize}
			\uncover<2->{\item	Extensive enhancement of the CLOCK algorithm: \emph{Clock with Adaptive Replacement} \cite{Bansal:2004}}
			\uncover<3->{\item	Approximation of the ARC page eviction strategy}
			\uncover<4->{\item	Uses two clocks and two LRU-lists}
			\uncover<5->{\item	Advantages of CAR compared to CLOCK:}
			\begin{itemize}
				\uncover<6->{\item	Weighted consideration of reference recency and frequency}
				\uncover<7->{\item	Scan-resistence}
			\end{itemize}
		\end{itemize}
	\end{block}

\end{frame}

\begin{frame}[fragile]

	\frametitle{CAR}
	
	\tikzset{%
		node distance = 1cm,
		refBit/.style = {draw = black, shape = rectangle},
		hand/.style = {very thick, draw = black},
		stack/.style = {rectangle split, rectangle split parts = #1, draw = black}
	}

	\begin{block}{Example}
		\centering
		\vspace{.5em}
		\begin{adjustbox}{totalheight = \textheight - 5.0em}
			\begin{tikzpicture}
				\node[]		(t1_center)	[]						{};
				\begin{scope}[start chain = circle placed {nodes around center = 45:12:t1_center:5em},
						      every node/.append style = {on chain = circle}]
					\foreach \cnt/\text in {0/1, 1/0, 2/1, 3/0, 4/1, 5/1, 6/0, 7/1, 8/0, 9/1, 10/0, 11/0}
						\node[refBit] (t1_\cnt) {\text};
					\chainin (circle-begin);
					\path[->]	
						(t1_center.center)		edge[hand]		(t1_10);
					\path[->]	
						($(t1_center.center)!0.75!(t1_10)$)		edge[bend left = 15]		($(t1_center.center)!0.75!(t1_9)$);
					\node[]		(t1_label)	[below = 6em of t1_center.center]	{\Large $T_1$};
					\node[]		(t1_head)	[right = -.125em of t1_10]	{\small head};
					\node[]		(t1_tail)	[right = -.125em of t1_11]	{\small tail};
					
					\node[stack = 20]	(b1)		[left = 6.5em of t1_center.center]	{};
					\node[]		(b1_label)	[below = .125em of b1]			{\Large $B_1$};
					\node[]		(b1_lru)	[right = -.125em of b1.one east]		{\small LRU};
					\node[]		(b1_mru)	[right = -.125em of b1.twenty east]	{\small MRU};
				\end{scope}
				
				\node[]		(t2_center)	[right = 6cm of t1_center]		{};
				\begin{scope}[start chain = circle placed {nodes around center = 45:24:t2_center:7.5em},
						      every node/.append style = {on chain = circle}]
					\foreach \cnt/\text in {0/1, 1/0, 2/1, 3/0, 4/1, 5/1, 6/0, 7/1, 8/0, 9/1, 10/0, 11/0, 12/1, 13/0, 14/1, 15/1, 16/0, 17/1, 18/0, 19/1, 20/1, 21/0, 22/1, 23/0}
						\node[refBit] (t2_\cnt) {\text};
					\chainin (circle-begin);
					\path[->]	
						(t2_center.center)		edge[hand]		(t2_12);
					\path[->]	
						($(t2_center.center)!0.8!(t2_12)$)		edge[bend left = 15]		($(t2_center.center)!0.8!(t2_10)$);
					\node[]		(t2_label)	[below = 8.5em of t2_center.center]	{\Large $T_2$};
					\node[]		(t2_head)	[left = -.125em of t2_12]	{\small head};
					\node[]		(t2_tail)	[left = -.125em of t2_13]	{\small tail};
					
					\node[stack = 16]	(b2)		[right = 9em of t2_center.center]	{};
					\node[]		(b2_label)	[below = .125em of b2]			{\Large $B_2$};
					\node[]		(b2_lru)	[left = -.125em of b2.one west]		{\small LRU};
					\node[]		(b2_mru)	[left = -.125em of b2.sixteen west]	{\small MRU};
				\end{scope}
			\end{tikzpicture}
		\end{adjustbox}
	\end{block}

\end{frame}

\subsection{Performance Evaluation}

\frame{\subsectionpage}

\begin{frame}[fragile]

	\frametitle{Buffer Pool Without Pointer Swizzling (TPC-C)}
	
	\tikzset{%
		DBSize/.style = { - , thick, dotted},
		DBSizeMark/.style = {draw = none, rotate = 90, anchor = south},
		latchedStyle/.style = {color = Maroon},
		gclockStyle/.style = {color = BurntOrange},
		carStyle/.style = {color = OliveGreen},
		latchedRate/.style = {latchedStyle, mark = x, only marks},
		gclockRate/.style = {gclockStyle, mark = x, only marks},
		carRate/.style = {carStyle, mark = x, only marks},
		DBSize/.append style = {draw = purple},
		DBSizeMark/.append style = {text = purple}
	}

	\centering
	\begin{adjustbox}{totalheight = \textheight - 2.0em}
		\begin{tikzpicture}[]
			\pgfplotsset{set layers}
			\begin{axis}[xlabel = {\ttfamily sm\_bufpoolsize $\left[\si{\gibi\byte}\right]$},
					   xlabel near ticks,
					   xmin = 0,
					   xmax = 20000,
					   xtick distance = {1000},
					   xticklabels = {,0,1,...,20},
					   scaled x ticks = false,
					   minor x tick num = 9,
					   ylabel = {$\text{average transaction throughput }\left[\sfrac{\text{transactions}}{\text{s}}\right]$},
					   ylabel near ticks,
					   ymin = 0,
					   ymax = 18000,
					   ymode = normal,
					   ytick distance = {2000},
					   scaled y ticks = false,
					   minor y tick num = 1,
					   axis y line* = left,
					   grid = major,
					   legend entries = {\small RANDOM, \small GCLOCK, \small CAR},
					   legend style = {at = {(-.2, -.175)}, anchor = north west, legend columns = -1, /tikz/every even column/.append style = {column sep = 0.025cm}},
					   scale only axis,
					   width = .875\textwidth,
					   height = .85\textheight]		
				\draw[DBSize]				({axis cs:13215, 0} |- {rel axis cs:0, 0})			--		({axis cs:13215, 0} |- {rel axis cs:0, 1});
				\node[DBSizeMark]			at ({axis cs:13215, 10000})		{initial database size};

				\addplot[visible on = <2->, latchedStyle, error bars/.cd, y dir = both, y explicit] table[x = buffersize, y = averagequerythroughput, y error = standarddeviationquerythroughput] {./tex/data/tpcc_noswizzling_latched.csv};
				\addplot[visible on = <3->, gclockStyle, error bars/.cd, y dir = both, y explicit] table[x = buffersize, y = averagequerythroughput, y error = standarddeviationquerythroughput] {./tex/data/tpcc_noswizzling_gclock.csv};
				\addplot[visible on = <4->, carStyle, error bars/.cd, y dir = both, y explicit] table[x = buffersize, y = averagequerythroughput, y error = standarddeviationquerythroughput] {./tex/data/tpcc_noswizzling_car.csv};
			\end{axis}

			\begin{axis}[xlabel = {\ttfamily sm\_bufpoolsize $\left[\si{\gibi\byte}\right]$},
					   xlabel near ticks,
					   xmin = 0,
					   xmax = 20000,
					   xtick distance = {1000},
					   xticklabels = {,0,1,..,20},
					   scaled x ticks = false,
					   minor x tick num = 9,
					   axis x line = none,
					   ylabel = {average hit rate},
					   ylabel near ticks,
					   ylabel style = {rotate = 180},
					   ymin = .65,
					   ymax = 1.02,
					   ymode = log,
					   scaled y ticks = false,
					   minor y tick num = 1,
					   axis y line* = right,
					   legend entries = {\small RANDOM, \small GCLOCK, \small CAR},
					   legend style = {at = {(1.2, -.175)}, anchor = north east, font = \scriptsize, legend columns = -1, /tikz/every even column/.append style = {column sep = 0.125cm}},
					   scale only axis,
					   width = .875\textwidth,
					   height = .85\textheight]		

				\addplot[visible on = <5->, latchedRate, error bars/.cd, y dir = both, y explicit] table[x = buffersize, y = averagehitrate, y error = standarddeviationhitrate] {./tex/data/tpcc_noswizzling_latched.csv};
				\addplot[visible on = <6->, gclockRate, error bars/.cd, y dir = both, y explicit] table[x = buffersize, y = averagehitrate, y error = standarddeviationhitrate] {./tex/data/tpcc_noswizzling_gclock.csv};
				\addplot[visible on = <7->, carRate, error bars/.cd, y dir = both, y explicit] table[x = buffersize, y = averagehitrate, y error = standarddeviationhitrate] {./tex/data/tpcc_noswizzling_car.csv};
			\end{axis}
		\end{tikzpicture}
	\end{adjustbox}
	
\end{frame}

\begin{frame}[fragile]

	\frametitle{Buffer Pool With Pointer Swizzling (TPC-C)}
	
	\tikzset{%
		DBSize/.style = { - , thick, dotted},
		DBSizeMark/.style = {draw = none, rotate = 90, anchor = south},
		latchedStyle/.style = {color = Maroon},
		gclockStyle/.style = {color = BurntOrange},
		carStyle/.style = {color = OliveGreen},
		latchedRate/.style = {latchedStyle, mark = x, only marks},
		gclockRate/.style = {gclockStyle, mark = x, only marks},
		carRate/.style = {carStyle, mark = x, only marks},
		DBSize/.append style = {draw = purple},
		DBSizeMark/.append style = {text = purple}
	}

	\centering
	\begin{adjustbox}{totalheight = \textheight - 2.0em}
		\begin{tikzpicture}[]
			\pgfplotsset{set layers}
			\begin{axis}[xlabel = {\ttfamily sm\_bufpoolsize $\left[\si{\gibi\byte}\right]$},
					   xlabel near ticks,
					   xmin = 0,
					   xmax = 20000,
					   xtick distance = {1000},
					   xticklabels = {,0,1,...,20},
					   scaled x ticks = false,
					   minor x tick num = 9,
					   ylabel = {$\text{average transaction throughput }\left[\sfrac{\text{transactions}}{\text{s}}\right]$},
					   ylabel near ticks,
					   ymin = 0,
					   ymax = 18000,
					   ymode = normal,
					   ytick distance = {2000},
					   scaled y ticks = false,
					   minor y tick num = 1,
					   axis y line* = left,
					   grid = major,
					   legend entries = {\small RANDOM, \small GCLOCK, \small CAR},
					   legend style = {at = {(-.2, -.175)}, anchor = north west, legend columns = -1, /tikz/every even column/.append style = {column sep = 0.025cm}},
					   scale only axis,
					   width = .875\textwidth,
					   height = .85\textheight]		
				\draw[DBSize]				({axis cs:13215, 0} |- {rel axis cs:0, 0})			--		({axis cs:13215, 0} |- {rel axis cs:0, 1});
				\node[DBSizeMark]			at ({axis cs:13215, 10000})		{initial database size};

				\addplot[visible on = <2->, latchedStyle, error bars/.cd, y dir = both, y explicit] table[x = buffersize, y = averagequerythroughput, y error = standarddeviationquerythroughput] {./tex/data/tpcc_swizzling_latched.csv};
				\addplot[visible on = <3->, gclockStyle, error bars/.cd, y dir = both, y explicit] table[x = buffersize, y = averagequerythroughput, y error = standarddeviationquerythroughput] {./tex/data/tpcc_swizzling_gclock.csv};
				\addplot[visible on = <4->, carStyle, error bars/.cd, y dir = both, y explicit] table[x = buffersize, y = averagequerythroughput, y error = standarddeviationquerythroughput] {./tex/data/tpcc_swizzling_car.csv};

			\end{axis}

			\begin{axis}[xlabel = {\ttfamily sm\_bufpoolsize $\left[\si{\gibi\byte}\right]$},
					   xlabel near ticks,
					   xmin = 0,
					   xmax = 20000,
					   xtick distance = {1000},
					   xticklabels = {,0,1,..,20},
					   scaled x ticks = false,
					   minor x tick num = 9,
					   axis x line = none,
					   ylabel = {$\text{average hit rate}$},
					   ylabel near ticks,
					   ylabel style = {rotate = 180},
					   ymin = .65,
					   ymax = 1.02,
					   ymode = log,
					   scaled y ticks = false,
					   minor y tick num = 1,
					   axis y line* = right,
					   legend entries = {\small RANDOM, \small GCLOCK, \small CAR},
					   legend style = {at = {(1.2, -.175)}, anchor = north east, font = \scriptsize, legend columns = -1, /tikz/every even column/.append style = {column sep = 0.125cm}},
					   scale only axis,
					   width = .875\textwidth,
					   height = .85\textheight]		

				\addplot[visible on = <5->, latchedRate, error bars/.cd, y dir = both, y explicit] table[x = buffersize, y = averagehitrate, y error = standarddeviationhitrate] {./tex/data/tpcc_swizzling_latched.csv};
				\addplot[visible on = <6->, gclockRate, error bars/.cd, y dir = both, y explicit] table[x = buffersize, y = averagehitrate, y error = standarddeviationhitrate] {./tex/data/tpcc_swizzling_gclock.csv};
				\addplot[visible on = <7->, carRate, error bars/.cd, y dir = both, y explicit] table[x = buffersize, y = averagehitrate, y error = standarddeviationhitrate] {./tex/data/tpcc_swizzling_car.csv};
			\end{axis}
		\end{tikzpicture}
	\end{adjustbox}
	
\end{frame}

\begin{frame}[fragile]

	\frametitle{Operation Performance}
	
	\pgfplotsset{%
		every axis/.append style = {
			ylabel near ticks,
			y label style = {font = \small},
			yticklabel style = {font = \tiny},
			ybar,
			xmin = -1.5,
			xmax = 1.5,
			xtick = {-1, 0, 1},
			xticklabels = {{RANDOM}, {GCLOCK}, {CAR}},
			x tick label style = {align = center, font = \footnotesize},
			xlabel near ticks,
			scale only axis,
			width = .95\textwidth,
			height = .85\textheight]		
		}
	}

	\tikzset{%
		noSwizzle/.style = {very thick},
		swizzle/.style = {very thick},
		noSwizzleTotal/.style = {very thick, ycomb},
		swizzleTotal/.style = {very thick, ycomb},
		noSwizzle/.append style = {draw = blue, fill = blue!50},
		swizzle/.append style = {draw = black, fill = black!25},
		noSwizzleTotal/.append style = {draw = blue, fill = blue!50, mark = *},
		swizzleTotal/.append style = {draw = black, fill = black!25, mark = *}
	}

	\centering
	\begin{adjustbox}{totalheight = \textheight - 2.0em}
		\begin{tikzpicture}
			\begin{axis}[ylabel = {$\text{average execution time }\left[\si{\nano\second}\right]$},
					  ymin = 0,
					  ymax = 2.25e+06,
					  ymode = normal,
					  scaled y ticks = false,
					  minor y tick num = 4,
					  axis y line* = left,
					  legend entries = {Traditional Buffer Pool, Pointer Swizzling Buffer Pool},
					  legend style = {at = {(-.15, -.175)}, anchor = west},
					  bar width = 4em,
					  ymajorgrids = true]
				\addplot[visible on = <2->, noSwizzle] coordinates
					{(-1, 52277.8) (0, 70276.2) (1, 1.97588e+06 + 19186)};
				\addplot[visible on = <4->, swizzle] coordinates
					{(-1, 34172.3) (0, 66239.2) (1, 1.95631e+06 + 18463.5)};
			\end{axis}
			\begin{axis}[ylabel = {$\text{total execution time }\left[\si{\second}\right]$},
					  ylabel style = {rotate = 180},
					  ymin = 0,
					  ymax = 600,
					  ymode = normal,
					  scaled y ticks = false,
					  minor y tick num = 1,
					  legend entries = {Traditional Buffer Pool, Pointer Swizzling Buffer Pool},
					  legend style = {at = {(1.15, -.175)}, anchor = east},
					  axis y line* = right]
				\addplot[visible on = <3->, noSwizzleTotal] coordinates
					{(-1.225, 549.645622) (-.225, 550.379224) (.775, 586.962186)};
				\addplot[visible on = <5->, swizzleTotal] coordinates
					{(-.775, 529.055048) (.225, 547.609491) (1.225, 585.685542)};
			\end{axis}
		\end{tikzpicture}
	\end{adjustbox}
	
\end{frame}

\subsection{Conclusion}

\frame{\subsectionpage}

\begin{frame}

	\frametitle{Conclusion}
	
	\begin{block}{\uncover<1->{Performance}}
		\begin{itemize}
			\uncover<2->{\item	CAR has a significantly higher hit rate than RANDOM or GCLOCK}
			\uncover<3->{\item	The hit rate of GCLOCK isn't significantly higher than the one of RANDOM}
			\uncover<4->{\item	Major differences in hit rate are only for buffer pool sizes of $\leq \frac{1}{10}$ of the database size}
			\uncover<5->{\item	The computational effort spent to do CAR eviction is \numrange{27}{58} times higher}
			\uncover<6->{\item	The overall performance of CAR isn't better than the one of RANDOM or GCLOCK}
		\end{itemize}
	\end{block}

\end{frame}


%------------------------------------------------
%\section*{References}
%------------------------------------------------

\begin{frame}[noframenumbering, allowframebreaks]{References}
	\printbibliography
\end{frame}

%------------------------------------------------
\section*{End}
%------------------------------------------------

\begin{frame}
\Huge{\centerline{Your Turn to Ask ...}}
\end{frame}

%----------------------------------------------------------------------------------------

\end{document} 